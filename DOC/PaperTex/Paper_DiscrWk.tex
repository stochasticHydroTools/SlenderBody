\subsubsection{Basis for $L^2$}
We will use a Chebyshev basis to discretize the fiber centerline. That is, 
\begin{equation}
\label{eq:basisD}
g_j(s) = \sum_{k=0}^{N-2} \alpha_{jk} T_k(s), 
\end{equation}
where $T_k(s)$ is the Chebyshev polynomial of the first kind of degree $k$ on $[0,L]$. 

The choice of $N-2$ for the maximum summation index is a necessary (but not sufficient) condition that makes the representation $\V{U}(s) = \M{K}\V{\alpha}$ unique on an $N$ point Chebyshev grid. To see this, suppose that Chebyshev polynomials of degree $N-1$ or higher were used in Eq.\ \eqref{eq:basisD}. Then the integration in Eq.\ \eqref{eq:du} results in Chebyshev polynomials of degree $N$ or higher contributing to the velocity. A degree $N$ polynomial can be zero at all $N$ nodes without being identically zero, and the representation is not unique. Thus the summation being from $k=0$ to $N-2$ in Eq.\ \eqref{eq:basisD} is a necessary condition for $\M{K}\V{\alpha}$ to give a unique representation of the fiber velocity. This is not a sufficient condition, however, since in practice $\V{n}_j$ is also a polynomial function of $s$ and $\int_0^s T_k(s') \V{n}_j(s') \, ds'$ could still be zero on a grid of size $N$. This means that $\M{K}\V{\alpha}$ could be zero with at least one nonzero $\alpha$ value, which means that $\V{\alpha}$ does not uniquely give $\V{U}$.

\subsubsection{Spatial discretization}

Because we use a collocation discretization, the fiber is discretized at nodes $s_i$, $i=1, \dots N$, where in our case $s_i$ is a node on a type 1 Chebyshev grid (i.e. a grid that does not include the endpoints; the reason for this choice is discussed in Section \ref{sec:rsc}). We then define an operator $\intmat$ that evaluates the integrals in Eq.\ \eqref{eq:du} with $\phi_k=T_k$ via some quadrature scheme. In the Chebyshev discretization, this must be done with proper anti-aliasing. In our discretization, the operator $\left(\intmat\left(T_k(\cdot) \V{n}_j(\cdot)\right)\right)(s)$ upsamples the functions $T_k$ and $\V{X}_s$ to a $2N$ grid. On the $2N$ grid, $\intmat$ then computes the normal vectors via Eq.\ \eqref{eq:nangles}, performs multiplication with $T_k$, and applies the pseudo-inverse of the Chebyshev differentiation matrix, $\M{D}_{2N}^\dagger$. $\intmat$ then downsamples this result from the $2N$ grid to the original $N$ grid. 

\subsubsection{Fully discrete linear system}
The spatially discrete form of Eq.\ \eqref{eq:du} is now given by
\begin{gather}
\label{eq:dvel}
\frac{\partial \bm{X}}{\partial t}\left(s_i\right)=  \left(\bm{M}(\bm{\lambda} +\bm{L}\bm{X})\right)(s_i) +\bm{U}_0(\bm{X}(s_i))= \left(\bm{K}\bm{\alpha}\right)(s_i) = \bm{U} +\sum_{j=1}^2\sum_{k=0}^{N-2} \alpha_{jk} \left(\intmat \left(T_k(\cdot ) \bm{n}_j(\cdot)\right)\right)(s_i)
\end{gather}

It is now straightforward to discretize the no work constraint in Eq.\ \eqref{eq:noworkcontL2}. We define a matrix $\bm{I}^*$ that integrates a (vector or scalar) function using Clenshaw-Curtis quadrature on the type 1 Chebyshev grid. Then the fully discrete form of Eq.\ \eqref{eq:noworkcontL2} is 
\begin{equation}
\label{eq:noworkcontFD}
\bm{K}^* \bm{\lambda}=\begin{pmatrix} \bm{I}^* \left(\left(\intmat(T_k(\cdot) \bm{n}_1(\cdot))\right) \cdot \bm{\lambda}\right) \\[2 pt] \bm{I}^* \left(\left(\intmat(T_k(\cdot) \bm{n}_2(\cdot))\right) \cdot \bm{\lambda}\right)\\[2 pt] \bm{I}^*\bm{\lambda} \end{pmatrix} = \begin{pmatrix} 0 \\[2 pt] 0\\[2 pt] \bm{0}\end{pmatrix},
\end{equation}
which must hold for all $k=0, 1, \dots, N-2$. 

Thus the final saddle-point system of equations for $\bm{\lambda}$ and $\bm{\alpha}$ (the discrete form of Eq.\ \eqref{eq:saddleL2}) can be written by combining Eqs.\ \eqref{eq:dvel} and \eqref{eq:noworkcontFD}, 
\begin{equation}
\label{eq:saddlept}
    \begin{pmatrix}
    -\bm{M} & \bm{K}\\[4 pt]
    \bm{K}^* & \bm{0}
    \end{pmatrix}
    \begin{pmatrix} 
    \bm{\lambda}\\[4 pt]
    \bm{\alpha}\\[4 pt]
    \end{pmatrix} =  \begin{pmatrix} 
    \bm{M}\bm{L}\bm{X}+\bm{U}_0\\[4 pt]
    \begin{pmatrix} \bm{0}\\[4 pt]
    -\bm{I}^* \bm{L}\bm{X} \end{pmatrix}
    \end{pmatrix}.
\end{equation}
This system, which has an obvious saddle-point structure, is not invertible generally because the representation $\bm{K}\bm{\alpha}$ is not necessarily unique. We therefore solve the system in the least squares sense with a tolerance of $10^{-6}$.

Note that in Eq.\ \eqref{eq:saddlept}, the matrices $\bm{M}$ and $\bm{K}$ are functions of $\bm{X}$. That is, $\bm{M}=\bm{M}(\bm{X})$ and $\bm{K}=\bm{K}(\bm{X})$. Finally, observe that in Eq.\ \eqref{eq:saddlept}, we enforce the third component of Eq.\ \eqref{eq:noworkcontFD} up to discretization errors in $\int \bm{f}^E(s) \, ds \approx \bm{I}^*\bm{f}^E=\bm{I}^*\M{L}\V{X}$. Although $\int \bm{f}^E(s) \, ds=\bm{0}$ in the continuous case, this does not necessarily hold discretely. We therefore keep the term $\bm{I}^*\M{L}\V{X}$ in our discretization to enforce the condition that the \textit{total force on the fiber is zero exactly in the discrete setting}.

%Our formulation of the problem is based on using the computed values of $\V{\alpha}$ to evolve $\V{X}_s$ in Eq.\ \eqref{eq:Xsupdate}, and then obtaining $\V{X}$ by integration (this entire process is Eq.\ \eqref{eq:defk}). This theory is of course predicated on the smoothness of $\V{n}_1(s)$ and $\V{n}_2(s)$. In Eq.\ \eqref{eq:nangles},. \cmt{(This paragraph is probably out of place.)}

\iffalse
\textit{Uniqueness of representation}. Next we show why the upper bound on the sum in Eq.\ \eqref{eq:basisD} is $N-2$ via studying the null space of the matrix $\bm{K}$. If $\bm{K}\bm{\alpha}(s_i)=\bm{0}$ for all $i = 1, \dots N$, we look for a condition that ensures $\bm{\alpha}=\bm{0}$. Without loss of generality, suppose that we choose $\bm{n}_1$ to have a zero entry at position $p$ and $\bm{n}_2$ to have a non-zero entry at position $p$. Then we can write the $p$th entry of $\bm{K}\bm{\alpha}$ at $s_i$ as 
\begin{align}
\left(\left(\bm{K}\bm{\alpha}\right)(s_i)\right)^p & = \bm{U} + \sum_{k=0}^{N-2} \bm{D}^+ \left(\alpha_{1k} T_k \cdot 0\right)(s_i) +  \bm{D}^+ \left(\alpha_{2k} T_k n_2^p\right)(s_i)\\[4 pt] & =\bm{U} + \sum_{k=0}^{N-2} \bm{D}^+ \left(\alpha_{2k} T_k n_2^p\right)(s_i) =0
\end{align}
where $n_2^p(s)$ is some nonzero function of $s$. So if $\left(\left(\bm{K}\bm{\alpha}\right)(s_i)\right)^p=0$ for all $i$, then $\left(\left(\bm{K}\bm{\alpha}\right)(s)\right)^p$ has $N$ zeros.  So a necessary condition for an empty null space of $\bm{K}$ is that $\left(\left(\bm{K}\bm{\alpha}\right)(s)\right)^p$ have $N-1$ zeros or less. Because of the integration operator $\bm{D}^+$, this means we can only include polynomials modes up to $N-2$. Note that this is a \textit{necessary} condition, not a suficient one, since in practice we cannot know the form of the normal vectors (those could be high order polynomials). 

As in the continuous case, since this equation must hold for every feasible motion $\bm{U}$ and $\bm{\alpha}$, each term in Eq.\ \eqref{eq:power} must be zero. Let us focus on the last term for now. Discretizing the integral using some quadrature rule at $s_i$ with weights $w_i$,
\begin{align}
    \mathcal{P}_2 &= \sum_{i=1}^{N} \sum_{j=1}^2\sum_{k=0}^{N-2} \alpha_{jk}\bm{D}^+\left(\phi_k \bm{n}_j\right)(s_i) \cdot \bm{\lambda}(s_i) w_i \\[4 pt]
& = \sum_{k=0}^{N-2} \sum_{j=1}^2 \alpha_{jk} \sum_{i=1}^{N} \bm{D}^+\left(\phi_k \bm{n_j}\right)(s_i) \cdot \bm{\lambda}(s_i) w_i=0, 
\end{align}
which must hold for any $\alpha_{jk}$. In particular, it must hold for $\alpha_{jk}=\delta_{1j} \alpha_k$ and $\alpha_{jk}=\delta_{2j} \alpha_k$, and so we have that \textit{for each} $j$ \textit{and} $k$, 
\begin{equation}
\label{eq:kstardisc}
    \sum_{i=1}^{N}\bm{D}^+\left(\bm{n}_j \phi_k\right)(s_i) \, \cdot \bm{\lambda}(s_i) w_i:=\bm{K}^*\bm{\lambda}=0.
\end{equation}
Eq.\ \eqref{eq:kstardisc} defines the bulk of the adjoint condition on $\bm{\lambda}$. It still remains to enforce the first part of Eq.\ \eqref{eq:power}. 

Since $\bm{U}$ is an arbitrary constant, the first term in Eq.\ \eqref{eq:power} can be discretized as
\begin{equation}
\label{eq:IT}
   \bm{0} = \int \bm{\lambda}(s) \, ds \approx \sum_{i=1}^{N} \bm{\lambda}(s_i) w_i := \bm{I}^*\bm{\lambda},
\end{equation}
where we have defined the discrete integration matrix $\bm{I}^*$ which takes a definite integral of a scalar function (whose values are given as a vector) on $[0,L]$. 
\fi

\subsubsection{Determining the elastic forces}
\label{sec:rsc}
The final order of business in the spatial discretization is to compute $\bm{f}^E=\M{L}\V X$ accurately and with the correct boundary conditions. We use rectangular spectral collocation \cite{tref17, dhale15} to determine the operator $\bm{L}$. We follow the convention of \cite{dhale15} and solve Eq.\ \eqref{eq:saddlept} on a type 1 Chebyshev grid with $N$ points (i.e. the grid where the PDE is posed \textit{does not} include the boundary points). The boundary conditions are imposed by upsampling the relevant quantities to a type 2 Chebyshev grid (that includes the endpoints) with $\tilde{N}=N+4$ points, since there are 4 BCs. %This procedure is analogous to using ghost cells in finite difference methods. %For the rest of this report, any quantity defined on the type 2 grid is marked with a tilde. 
Given the values of $\bm{X}$ on a type 1 Chebyshev grid with $N$ points, there is a unique configuration $\tilde{\bm{X}}$ on the type 2 grid that satisfies
\begin{equation}
\label{eq:deftilde}
\begin{pmatrix} \bm{R} \\[2 pt] \bm{B} \end{pmatrix} \tilde{\bm{X}} = \begin{pmatrix} \bm{I}_N \\[2 pt] \bm{0} \end{pmatrix} \bm{X}. 
\end{equation}
Here $\bm{R}$ is the downsampling operator that intepolates the data on the type 1 $N$ point grid from the data on an $\tilde{N}=N+4$ type 2 grid, and $\bm{B}$ is the operator that encodes the boundary conditions $\displaystyle \tilde{\bm{X}}_{ss}\left(s=0,L\right)=\tilde{\bm{X}}_{sss}\left(s=0,L\right)$ on the \textit{type 2 grid} (note that the right hand side of Eq.\ \eqref{eq:deftilde} could easily be modified to encode other boundary conditions). Because $\M{R}$ is an $N \times (N+4)$ matrix and $\M{B}$ is a $4 \times (N+4)$ matrix, the left hand side of Eq.\ \eqref{eq:deftilde} is invertible and we can therefore write 
\begin{equation}
\label{eq:getX}
\tilde{\bm{X}} = \begin{pmatrix} \bm{R} \\[2 pt] \bm{B} \end{pmatrix}^{-1} \begin{pmatrix} \bm{I}_N \\[2 pt] \bm{0} \end{pmatrix} \bm{X}= \bm{E}\bm{X}.
\end{equation}

Thus for every configuration $\bm{X}$, there is a unique function $\tilde{\bm{X}}$ on the type 2 grid that satisfies Eq.\ \eqref{eq:getX}.  In finite difference schemes, there are unique values of the ``ghost cells'' that allow the boundary stencils to satisfy the BCs to some order. Thus the rectangular spectral collocation method can be thought of as an extension of ghost cell techniques for finite difference methods.  

The function $\tilde{\bm{X}}$ can be used to compute $\bm{f}^E$ in a way consistent with the boundary conditions. That is, $\tilde{\bm{f}}^E=-E\tilde{\bm{X}}_{ssss}$ is computed on the \textit{type 2} grid and then downsampled via the operator $\bm{R}$ (analogous to using ghost cells at the boundaries to compute the fourth derivative of $\bm{X}$). We write the downsampled bending force as
\begin{equation}
\label{eq:fE}
\bm{f}^E=\bm{R}\tilde{\bm{f}}^E=-\bm{R}E\tilde{\bm{D}}^4 \tilde{\bm{X}} = -E\bm{R}\tilde{\bm{D}}^4 \bm{E}\bm{X}:=\bm{L}\bm{X}.
\end{equation}
So that we have defined the bending force on the type 1 grid, $\bm{f}^E=\bm{L}\bm{X}$. Notice that $\bm{L}$ is a \textit{constant} matrix (not a function of $\bm{X}$). It can therefore be precomputed and applied to compute $\V{f}^E$ for a given configuration $\V{X}$. 
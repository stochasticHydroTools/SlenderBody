In this section, we compare the output of our algorithm directly with that of \cite{ehssan17} for a fiber with initial tangent vector
\begin{equation}
\label{eq:Xst0}
\bm{X}_s(s,t=0) = \frac{1}{\sqrt{2}}\begin{pmatrix} \cos{\left(s^3 (s-L)^3\right)}\\[2 pt] \sin{\left(s^3(s-L)^3\right)}\\[2 pt] 1 \end{pmatrix}. 
\end{equation}
This choice of tangent vector satisfies the boundary conditions $\displaystyle \bm{X}_{ss}\left(s=0,L\right)=\bm{X}_{sss}\left(s=0,L\right)=\bm{0}$ and the inextensibility constraint. Because there is no analytical solution for its configuration, we obtain $\V{X}(t=0)$ by integrating Eq.\ \eqref{eq:Xst0} on the $N$ point grid using the Chebyshev integration matrix \cite{greengard1991spectral}. We do this, as opposed to integrating Eq.\ \eqref{eq:Xst0} to machine precision numerically, so that the Chebyshev expansions on an $N$ point grid of $\V{X}_s$ and $\V{X}$ are consistent at $t=0$. Because the algorithm of \cite{ehssan17} updates $\V{X}$ rather than $\V{X}_s$, they proceed in the latter fashion, and so much of the spcatial differences between the two algorithms will be visible even at $t=0$. 

Beginning with the tangent vector in Eq.\ \eqref{eq:Xst0}, we simulate the fiber relaxation until $t_f=0.01$, using $E=\mu=1$ for simplicity and $\epsilon=10^{-3}$. For this test only, we use ellipsodial fibers to facilitate comparison with \cite{ehssan17}, so that $c(s)=-\text{log}(\epsilon^2)$ in Eq.\ \eqref{eq:fibevcont}. 

In general, we will use an $L^2$ function norm to compute the differences between configurations throughout this section. Given two configurations $\bm{X}$ and $\bm{Y}$, the $L^2$ norm of their difference is defined as
\begin{equation}
\label{eq:Errort}
E[\bm{X},\bm{Y}](t) = \left(\int_0^L \norm{\bm{X}(t)-\bm{Y}(t)}^2 \, ds\right)^{1/2} = \left( \sum_{i=1}^{1000} \norm{\bm{X}(s_i,t)-\bm{Y}(s_i,t)}^2 \, w_i  \right)^{1/2}. 
\end{equation}
The final term denotes the fact that we upsample each configuration to a common type 2 grid of $1000$ points to measure the error. 

\subsubsection{Spatial error}
We begin with an analysis of the spatial errors in both algorithms. To isolate the spatial errors, we set $\Delta t =10^{-6}$, so that a given spatial discretization is temporally converged to at least 7 digits. We use as an exact solution the result of \cite{ehssan17} (the strong formulation) when $N=24$.   

Fig.\ \ref{fig:locspatial} shows the $L^2$ error over time for several different spatial discretizations. We observe first that our weak formulation errors are relatively constant throughout the simulation, and are generally maximal when $t=0$. This stems from our Chebyshev integration of the tangent vectors to obtain the positions, rather than the exact integration used in the strong formulation simulations. Considering now the maximal $L^2$ error over time, Fig.\ \ref{fig:locspatial} clearly shows that our discretization vastly improves spatial accuracy over that of \cite{ehssan17}. In particular, only $N=4$ points in our weak formulation are required to obtain the same spatial error as $N=12$ points in the strong formulation of \cite{ehssan17}. Furthermore, we obtain spatial errors below $10^{-4}$ when $N=12$ points are used, so that our results are $100$ times more accurate for the same amount of work. 


\begin{figure}
\centering 
\includegraphics[width=70mm]{LocalFigs/SpatialLocalPAPER.eps}
\caption{\label{fig:locspatial}Spatial errors in the relaxing fiber simulation. The $L^2$ norm error in position over time is shown for $N=4$ (dashed-dotted blue), $N=8$ (dotted red), and $N=12$ (solid yellow). The norm is defined in Eq.\ \eqref{eq:Errort}. We compare our results to the strong formulation when $N=12$ (dashed purple) and find that we obtain errors $\approx 100$ times smaller when $N=12$. } 
\end{figure}

\subsubsection{Temporal error}
We next consider the temporal error in our algorithm by using the trajectory with $\Delta t = 10^{-6}$ as an exact solution. We then simulate the fiber trajectory with timesteps $2, 4, 8, $ and $16 \times 10^{-6}$ and record the maximal $L^2$ error in time using Eq.\ \eqref{eq:Errort}. The results are given in Fig.\ \ref{fig:loctemporal} for $N=12$ and $N=24$. We see second-order convergence in time for both spatial discretizations. This is an improvement on the first-order discretization used in \cite{ehssan17}. 

\begin{figure}
\centering
\subfigure[$N=12$]{ 
\includegraphics[width=70mm]{LocalFigs/TemporalLocalPAPER.eps}}
\caption{\label{fig:loctemporal}Temporal errors in local drag simulation. For each spatial discretization, we use $\Delta t=10^{-6}$ as an exact solution for and measure the maximum $L^2$  error in position (defined in Eq.\ \eqref{eq:Errort}) over time. Using $\Delta t = 2, 4, 8, 16 \times 10^{-6}$ and $N=12$ (blue circles) and $N=24$ (orange squares), we observe second-order convergence in time. \cmt{Empirical orders from largest $\Delta t$ to smallest for $N=12$: 1.69, 2.20, 1.72, and for $N=24$: 1.93, 1.96, 2.10. Put these in a table?} } 
\end{figure}
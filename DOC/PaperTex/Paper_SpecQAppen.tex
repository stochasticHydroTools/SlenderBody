The goal of this appendix is to provide some rationale about our choice of discretization for the special quadrature scheme. Following this, we also discuss the creation of a continuous velocity field that is similar to the approach taken in \cite{ts04}. 

\subsection{Choice of special quadrature discretization}
Given the novelty of the special quadrature scheme, we test its accuracy here in our context. To do this, we generate 100 different inextensbile filaments whose tangent vectors and force densities are represented by the same exponentially decaying Chebyshev series with $N=16$ coefficients (the decay is such that the last coefficient has value at most $10^{-4}$). We place 100 targets a distance $d$ in the normal direction from the fiber centerlines and compute the integrals in Eq.\ \eqref{eq:jf2} by direct quadrature to 10 digits and also using the special quadrature scheme of \cite{barLud} with varying fiber centerline discretizations. Our goal is to obtain 3 digits of accuracy irrespective of the fiber shape and distance from target to fiber. 

\subsubsection{Accuracy of direct quadrature}
Our first objective is to determine when to call the special quadrature routine, i.e. when the accuracy of direct quadrature breaks down. To do this, we non-dimensionalize the distance $d$ by the fiber length $L$, so that the targets are positioned a fixed $d/L$ from the fiber. We then integrate Eq.\ \eqref{eq:jf1} directly with $N=16$ and $N=32$ points. As shown in Fig.\ \ref{fig:dirquad}, the boundary at which we start getting less than 3 digits of accuracy is $d/L \approx 0.15$ for $N=16$ and $d/L \approx 0.06$ for $N=32$. So we need some form of special quadrature below $d/L=0.15$ when $N=16$ and below $d/L=0.06$ when $N=32$. 

\begin{figure}
\centering
\subfigure[$N=16$]{
\includegraphics[width=70mm]{LudQuadFigs/dirNL16.eps}
\includegraphics[width=70mm]{LudQuadFigs/dirNL16_ds.eps}}
\subfigure[$N=32$]{
\includegraphics[width=70mm]{LudQuadFigs/dirNL32.eps}
\includegraphics[width=70mm]{LudQuadFigs/dirNL32_ds.eps}}
\caption{Relative errors using direct quadrature for the slender body kernel of Eq.\ \eqref{eq:jf1}. We consider (a) $N=16$ and (b) $N=32$ points and vary the distance $d/L$ of the targets from the fiber. We see that we obtain 3 digits of accuracy using $N=16$ when $d/L \geq 0.15$. For $N=32$, the 3 digit threshold occurs around $d/L = 0.06$. }
\label{fig:dirquad}
\end{figure}

\subsubsection{Accuracy of special quadrature}
Given that special quadrature must be called at a fixed $d/L$, the question of what discretization to use remains. The need for a finer discretization at small distances arises because the function $\phi(\eta)$ in Eq.\ \eqref{eq:phi} becomes less smooth (and therefore less well-represented by a monomial basis) as $\V{x}$ approaches the fiber centerline. In our evaluation of special quadrature accuracy, we non-dimensionalize by the fiber radius, so that we are concerned with small values of $d/r=d/(\epsilon L)$. 

Fig.\ \ref{fig:specquad} clearly shows that we need to use 2 panels of 32 points when $d/\epsilon L < 8$. When $d/\epsilon L \geq 8$, we obtain 3 digits of relative accuracy using 1 panel of 32 points along the fiber. The transition to 1 panel of 16 points occurs at larger distances, specifically $d/\epsilon L \geq 60$ allows for 1 panel of 16 points to give 3 digits. The vital point here is that $d/\epsilon L \gg 1$ when we can use 1 panel of 16 points, and so at this large distance we are simply integrating the Stokeslet, as the doublet correction is insignificant. This means that the relevant non-dimensional distance is $d/L=0.06$. So the distance at which 1 panel of 16 points gives 3 digits using special quadrature is the same as the distance at which 1 panel of 32 points gives 3 digits using direct quadrature. This means that we can bypass special quadrature with 1 panel of 16 points and instead simply do the integral directly with 32 points for $0.15 > d/L > 0.06$. 

\begin{figure}
\centering
\subfigure[$d/\epsilon L=2$]{\includegraphics[width=0.3\textwidth]{LudQuadFigs/d4_2pan32.eps}}
\subfigure[$d/\epsilon L=8$]{\includegraphics[width=0.3\textwidth]{LudQuadFigs/d16_1pan32.eps}}
\subfigure[$d/\epsilon L=64$]{\includegraphics[width=0.3\textwidth]{LudQuadFigs/d60_1pan16.eps}}
\caption{Integrating the slender body kernel with $L=2$ and $\epsilon=10^{-3}$. We consider 100 fibers with 100 targets per fiber and plot the relative error over the targets for (a) $d/\epsilon L = 2$, (b) $d/\epsilon L = 8$, and (c) $d/\epsilon L = 60$, $d/L = 0.06$ (here since $d/\epsilon L \gg 1$, $d/L$ is the relevant non-dimensional distance). The independent variable in is the mean relative curvature on the fiber, $\norm{\bm{X}_{ss}}/C_{circ}$, where $C_{circ} =2\pi/(L\sqrt{2})$ is the curvature of a circular fiber with the same arclength. }
\label{fig:specquad}
\end{figure}

\subsection{Estimating distance from the fiber}
Having established that the spatial discretization needed to achieve 3 digits depends on the distance of the target from the fiber, we next define a measure of distance from the centerline. 

\subsubsection{Long distances}
Let us consider first long distances. When we are far from the fiber, we can sample the fiber locations at $N_u=16$ uniform points and compute the distance to the fiber as the discrete minimum over these points. 

Fig.\ \ref{fig:unifders} shows the relative errors in the fiber distance computation using $N_u=16$ uniform points along the fiber. At $d/L = 0.15$, we observe an over-estimation of at most 5\% in the distance using uniform points. For $d/L=0.06$, the over-estimation of at most 15\% in the distance is predictably larger (for comparison, using 16 Chebyshev points gives an error as large as 30\%). So for $N_u=16$, we conclude that we can estimate to 5\% accuracy whether or not a target is within $d/L=0.15$ of the fiber using the uniform point scheme. Likewise for $d/L=0.06$, we can estimate the distance to about 15-20\% accuracy. This means that we can use the uniform point sampling procedure to tell us whether to call the special quadrature routine. 

\subsubsection{Short distances}
Once we determine the special quadrature scheme is needed, we need to use a different approximation for distances close to the fiber that determines when to change the number of panels. Consider the root $\eta^*$ determined from the near-singular quadrature scheme with 1 panel. We write 
\begin{equation}
\label{eq:eta}
\eta_r=\begin{cases} \text{Re}(\eta^*) & -1 \leq \text{Re}(\eta^*) \leq 1\\[2 pt] -1 & \text{Re}(\eta^*) < -1 \\[2 pt]1 &  \text{Re}(\eta^*) > 1 \end{cases}
\end{equation} 
as an approximation to the rescaled (on $[-1,1]$) $\eta$ coordinate on the fiber centerline of the closest point. If $\bm{g}(\eta)=\bm{g}_r(\eta)+i\bm{g}_i(\eta)$ is the representation of the fiber centerline, extended analytically to the complex plane, then the approximate shortest distance to the fiber is $d_r=\norm{\bm{x}-\bm{g}_r(\eta_r)}$. Note that this gives an approximation to the distance even for points that are off the fiber in the tangential direction. 

Fig.\ \ref{fig:grters} shows that when $d/\epsilon L \leq 8$, we make at most a 10\% error in using $d_r$ as the normal distance from the fiber. Thus we can use $d_r$ to determine whether we need 1 panel of 32, 2 panels of 32, or if the distance is less than $2\epsilon L$ and special approximations need to be performed. 

\begin{figure}
\centering
\subfigure[Large distances - evenly spaced points]{\label{fig:unifders}
\includegraphics[width=0.45\textwidth]{LudQuadFigs/dUnifN16.eps}
\includegraphics[width=0.45\textwidth]{LudQuadFigs/dUnifN32.eps}}
\subfigure[Special quadrature]{\label{fig:grters}
\includegraphics[width=0.45\textwidth]{LudQuadFigs/groots_d4m3.eps}
\includegraphics[width=0.45\textwidth]{LudQuadFigs/groots_d16m3.eps}}
\caption{Computing the normal distances from the fiber two different ways. (a) Relative errors in computing the distance for $d/L = 0.15$ (left) and $d/L= 0.06$ (center) by resampling the curve at $N_u=16$ uniformly spaced points and replacing the continuous minimization problem with a discrete one. (b) The relative error in computing the distance using $d_r=\norm{\bm{x}-\bm{g}_r(t_r)}$ from the special quadrature scheme. The error is at most 10\% of the distance for $d/\epsilon L = 2, 8$. \cmt{Plots in (b) are very uninformative. Replace with box plots? those would show the outliers better.}  }
\end{figure}

\subsection{Continuous velocity field \label{sec:contvel}}
Given a target $\V{x}$ and fiber centerline $\V{X}(s)$, we have established that we can estimate the minimum distance $d$ from $\V{x}$ to $\V{X}(s)$. We recall that the integral in Eq.\ \eqref{eq:jf1} only makes physical sense when $d > 2\epsilon L$, as otherwise the ``cross sections'' of the target fiber and the source fiber overlap. Therefore if $d \leq 2 \epsilon L$, we set the velocity at $\V{x}$ to be exactly equal to the centerline velocity on the closest point $\V{X}(\eta_r)$, where $\eta_r$ is defined in Eq.\ \eqref{eq:eta}. As a consequence of the asymptotics of SBT which deal with a pointwise, rather than averaged, flow field, this velocity does not coincide exactly with the integral of Eq.\ \eqref{eq:jf1} at $d=2\epsilon L$. Because of this, we set the velocity to be equal to that given in Eq.\ \eqref{eq:jf1} only when $d > 4\epsilon L$. Between $d=2\epsilon L$ and $d=4 \epsilon L$, we linearly interpolate the velocity between the centerline velocity and the integrand in Eq.\ \eqref{eq:jf1}, which can be computed to controlled accuracy using special quadrature. This linear interpolation technique is similar to that of \cite{ts04}, except here it is dependent on the problem physics and not the spatial discretization. 

The tests of this section lead naturally to the ``near fiber'' algorithm in Fig.\ \ref{fig:algflow}. Note that this algorithm includes the error estimates from the calculation of distances. We assume that there are at least $N=16$ points on the fiber, so that when $d/L \geq 0.15$, the velocity computed from direct quadrature without upsampling is accurate to at least 3 digits. 

\begin{figure}[ht]
\centering
\includegraphics[width=0.8\textwidth]{LudQuadFigs/FlowChart.png}
\caption{Overall near field algorithm, where $d^*=d_r/(\epsilon L)$. We assume that there are at least $N=16$ points on the fiber, so that when $d/L \geq 0.15$, the velocity computed from direct quadrature without upsampling is accurate to at least 3 digits. }
\label{fig:algflow}
\end{figure}
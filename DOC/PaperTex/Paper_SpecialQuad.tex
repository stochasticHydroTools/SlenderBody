The contribution of fiber $j$ to the velocity at point $s_i$ on fiber $i$ is given by the line integral
\begin{gather}
\label{eq:jf1}
\V{U}_{JF}(s_i) =\int_0^L \left(\frac{\M{I}+\hat{\V{R}}(s_i,s_j)\hat{\V{R}}(s_i,s_j)}{\norm{\V{R}(s_i,s_j)}} + (\epsilon L)^2 \frac{\M{I}-3\hat{\V{R}}(s_i,s_j)\hat{\V{R}}(s_i,s_j)}{\norm{\V{R}(s_i,s_j)}^3}\right)\V{f}^j(s_j) \, ds_j. 
\end{gather}
When $\V{X}^i(s_i)$ (the \textit{target point} on fiber $i$) approaches the centerline $\V{X}^j$ of fiber $j$, this integral becomes \textit{nearly singular}, and again special quadrature schemes are needed to evaluate it accurately. We begin by rewriting Eq.\ \eqref{eq:jf1} so that the near-singularity is entirely in the denominator (i.e. by removing the hats)
\begin{gather}
\label{eq:jf2}
\V{U}_{JF}(s_i) =\int_0^L \left(\frac{\M{I}}{\norm{\V{R}(s_i,s_j)}}+\frac{\V{R}(s_i,s_j)\V{R}(s_i,s_j)+(\epsilon L)^2 \M{I}}{\norm{\V{R}(s_i,s_j)}^3} -3(\epsilon L)^2 \frac{\V{R}(s_i,s_j)\V{R}(s_i,s_j)}{\norm{\V{R}(s_i,s_j)}^5}\right)\V{f}^j(s_j) \, ds_j. 
\end{gather}
Here we briefly summarize the scheme developed recently in \cite[Section~3]{barLud} to compute this integral with controlled accuracy. If we once again apply the rescaling $\eta = -1+\frac{2}{L}s_j$, then Eq.\ \eqref{eq:jf2} shows that the integrals we need to evaluate are of the form
\begin{equation}
\label{eq:nsing}
I(\V{y}) = \frac{L}{2}\int_{-1}^1 \frac{f(\eta)}{\norm{\V{X}^j(\eta) -\V{x}}^m} \, d\eta
\end{equation}
for $m=1, 3, 5$ and a smooth density $f(\eta)$. Now, the idea of \cite{barLud} is to extend the representation of $\V{X}^j(\eta)$ to the \textit{complex} plane and compute the complex root of $\norm{\V{X}^j\left(\eta^*\right)-\V{x}}=0$. Because the centerline representation $\V{X}^j(\eta)$ is available as a Chebyshev series, it is simple to solve for the root $\eta^*$ via Newton iteration. 

Once the root is known, the singularity can be removed from the integrand by rewriting Eq.\ \eqref{eq:nsing} as
\begin{equation}
\label{eq:intfactored}
I(\V{y}) = \frac{L}{2}\int_{-1}^1 \frac{f(\eta)((\eta-\eta^*)\overline{(\eta-\eta^*)})^{m/2}}{\norm{\V{X}^j(\eta) -\V{x}}^m} \frac{1}{((\eta-\eta^*)\overline{(\eta-\eta^*)})^{m/2}}\, d\eta, 
\end{equation}
so that the integral is written as a smooth function times a near singular function. Note the analogy with the finite part case, Eq.\ \eqref{eq:Jrewrite}, where the integrand is again written as the product of a smooth function times a singular function. In addition, the Bernstein radius of the root $\eta^*$ can be used to determine whether direct quadrature with $N$ points will suffice to compute the integral in Eq.\ \eqref{eq:intfactored} for any given tolerance \cite{barLud}.

The procedure of Section \ref{sec:tornFP} can be repeated to compute the integral via a monomial expansion. Let us define 
\begin{equation}
\label{eq:phi}
\phi(\eta)  = \frac{f(\eta)((\eta-\eta^*)\overline{(\eta-\eta^*)})^{m/2}}{\norm{\V{X}^j(\eta) -\V{x}}^m}
\end{equation}
and use the adjoint method of Eq.\ \eqref{eq:specscheme} to compute monomial coefficients of $\phi(\eta)$. The integrals 
\begin{equation}
q_k(\eta^*) = \frac{L}{2}\int_{-1}^1 \eta^k \frac{1}{((\eta-\eta^*)\overline{(\eta-\eta^*)})^{m/2}}\, d\eta,
\end{equation}
are then computable exactly by recurrence relations, as discussed in \cite[Section~3.1]{barLud}. Thus the entire accuracy of the scheme depends on representing $\phi(\eta)$ in Eq.\ \eqref{eq:phi} in a monomial basis accurately. Because of the scarcity of information on this, we have conducted our own tests in Appendix \ref{sec:nearfibtests}. These tests lead naturally to a method to compute the integral in Eq.\ \eqref{eq:jf1} to a guaranteed 3 digits of accuracy for any target and fiber. 

Appendix \ref{sec:nearfibtests} also shows how we can estimate the distance between a given target and fiber centerline to within 10\% accuracy. Suppose now that this distance, defined as $d$, is known exactly. Because each point on the fiber centerline is positioned inside a cross section of radius $r=\epsilon L$, the integral in Eq.\ \eqref{eq:jf1} only makes physical sense when $d > 2\epsilon L$ (when the ``cross sections'' are not in contact). In order to establish a smooth velocity field, we draw an analogy with \cite{ts04} to establish a continuous velocity field when $d=\mathcal{O}(\epsilon L)$. This choice is essentially arbitrary and is important solely to make the nonlocal mobility $\M{M}_{NL}$ continuous. Because of this, we have relegated it to Appendix \ref{sec:contvel}. 
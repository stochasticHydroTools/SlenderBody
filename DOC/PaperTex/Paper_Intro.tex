This paper is concerned with the interactions of long, thin, inextensible filaments with a viscous fluid. Examples of these interactions abound in biology, engineering, physics, and medicine. In biology, the swimming mechanisms of flagellated organisms have been of interest for decades, with initial focus on using force and torque balances to understand swimming mechanics \cite{chwang1971note, berg1973bacteria, brennen1977fluid, lauga2009hydrodynamics}, and more recent studies on flagellar bundling \cite{lim2012fluid, maier2016magnetic}. In physics and engineering, suspensions of high-aspect-ratio fibers have been observed to display non-Newtonian, viscoelastic behavior both experimentally \cite{fibexps} and computationally \cite{mackfibs, kochsumfig}. 

Our particular area of interest is the simulation of filaments that make up the cell cytoskeleton. These filaments, which include microtubules and actin filaments and have aspect ratios from 100 to 10,000, maintain the cell's structure and control the mechanics of the cell division process \cite{alberts}. In vivo, actin filaments are generally bound together into networks by cross-linking proteins, the properties of which determine the viscoelastic behavior of the cytoskeleton \cite{wagner2006cytoskeletal, head2003deformation, ahmed2015dynamic}. While much of the recent work on slender body hydrodynamics has been in the context of microtubules and the positioning of the cellular spindle \cite{ehssan17, shelley2016dynamics}, to our knowledge there has been no systematic study of the influence of hydrodynamic solvent interactions on the mechanics of cross-linked actin networks. One of the goals of this paper is therefore to develop an efficient numerical technique for the simulation of $\mathcal{O}(100-1000)$ cross-linked actin networks, taking into account their interactions with a viscous (zero Reynolds number) solvent. 

Prior to the year 2000, tools for analytical analysis and numerical simulation of filaments in Stokes flow were generally developed in parallel, but separately. For slender filaments, a useful approach for both analysis and computation is to reduce the problem from three dimensions to one by assuming a certain distribution of singularities along the filament centerline. This approach, referred to as ``slender body theory'' (SBT), was first introduced by Hancock \cite{hancock1953self} and later expanded upon by Batchelor \cite{batchelor1970slender}. By using the method of matched asymptotics, Keller and Rubinow were the first to derive an SBT that is uniformly accurate in the fiber slenderness ratio $\epsilon = $ radius/length. Johnson further developed the theory by adding higher order corrections and correctly treating a fiber with free ends \cite{johnson}, and G\"otz re-derived the SBT of Keller and Rubinow in a more general context, allowing him to apply the theory to Oseen's and Poisson's equations \cite{gotz2001interactions}. 

Because the SBTs of Keller and Rubinow, Johnson, and G\"otz are uniformly accurate in powers of $\epsilon$, they have formed the basis of most modern analysis. To this end, Mori \textit{et al.} recently showed that these singularity solutions solve a well-posed Stokes problem with non-standard boundary conditions on the filament surface \cite{mori2018theoretical, morifree}. Koens and Lauga also showed that the SBT singularity solution can be recovered by matched asymptotic expansion of the full surface boundary integral formulation of Stokes flow \cite{koens2018boundary}. 

On the numerical side, non-SBT based techniques for the simulation of fibers in Stokes flow have been in use for many decades. The most prevelant among these are so-called immersed boundary methods, in which the fibers are discretized by a series of marker points, each of which is assigned a force according to the fiber physics. In one approach, the force is regularized by spreading to a background fluid grid using a compactly supported regularized delta function. The fluid velocity is then found by solving the fluid equations on the grid, and interpolation of this velocity field back onto the marker points yields the marker velocities (this is the IB method of Peskin and collaborators \cite{peskin1972flow, peskin2002acta}). In a similar yet distinct approach, the force can also be regularized using some Gaussian-like blob or cutoff function, and the Stokes equations can be solved analytically for that forcing to obtain the marker velocities (the method of regularized Stokeslets of Cortez and collaborators \cite{cortez2001method, cortez2005method}). 

In both of these methods, proper modeling of slender fibers requires that the width of the regularization function be on the order of the fiber radius \cite{ttbring08}. Since the regularization width is also tied to the fiber discretization spacing (and, in the IB method, the fluid grid spacing), slender fibers must be discretized with many more points than would be necessary in a continuum, SBT-based approach. While this limitation can be partially overcome with adaptive mesh refinement \cite{griffith2007adaptive}, grid coarsening with local velocity correction \cite{maxian19}, and kernel independent fast multipole methods (to accelerate many-body sums) \cite{rostami2016kernel}, the fact remains that to achieve controlled accuracy for dilute suspensions of many fibers, IB methods are generally more expensive than continuum-based approaches. 

Recognizing this, Shelley and Ueda were the first to apply a continuum-based SBT approach to the immersed slender fiber problem. By designing a numerical method around the analytical results of slender body theory, they reduced the complexity of the numerical computations from three dimensions to one \cite{shelley1996nonlocal, shelley2000stokesian}. Their formulation, however, relies on the filament being a closed loop, thus excluding many problems from biology, engineering, and physics where the filament ends are free. 

Tornberg and Shelley treated inextensible filaments with free ends using an SBT-based numerical method \cite{ts04}. In their approach, inextensibility is preserved by deriving an auxiliary integro-differential equation for the line tension in the filament, which acts as a Lagrange multiplier to preserve inextensibility. The line tension PDE combined with the SBT evolution equation for the fiber position gives a closed system for the filament dynamics. The method of \cite{ts04} has since been used in applications with flexible (and sometimes fluctuating) filaments  \cite{manikantan2013subdiffusive, young2009hydrodynamic}, and was also extended to simulate falling rigid fibers, the novelty there being that many of the SBT-related integrals can be done analytically \cite{tornberg2006numerical}. More recently, Nazockdast \textit{et al.} modified the approach of Tornberg and Shelley to make it feasible to simulate many-body cellular fiber assemblies. By replacing the second-order spatial discretization of Tornberg and Shelley with a spectral spatial discretization and utilizing a kernel-independent FMM to accelerate sums, they developed a parallel algorithm that makes it possible to simulate $\mathcal{O}(1000)$ fibers in linear time \cite{ehssan17}. 

Despite these recent advances, a need still exists for a re-examination of SBT-based approaches for inextensible fibers. For example, the line tension equation of \cite{ts04} involves multiplications of high-order (as high as four) derivatives of the fiber position function. Thus in the spectral formulation of \cite{ehssan17}, aliasing results, and a loss of accuracy occurs in the spatial discretization. In addition, the ``inextensibility'' of the filaments is still subject to discretization error and requires inserting a penalty term into the line tension equation that reduces the discrete extensibility \cite{ts04}. For fibers tugged by cross-linkers, this penalty parameter will be large, introducing artificial stiffness into the problem. And finally, the most recent contribution of \cite{ehssan17} neglects higher-order corrections in the SBT equations when filaments are close together, leading to a loss of accuracy when filaments approach each other. 

The primary focus of this paper is on a new formulation for inextensible filaments. In our approach, the fibers are evolved via a rotation of the tangent vector on the unit sphere, and the new positions are then obtained by integration. In this way, we maintain exact inextensibility of the fibers without introducing a penalty parameter. We couple this advance with the latest techniques \cite{barLud, tornquad} for efficient evaluation of each component in nonlocal slender body theory to present a method with improved accuracy and robustness over that of \cite{ehssan17}. 

The rest of this paper is laid out as follows: we begin in Section 2 by introducing the necessary SBT equations for both the local and nonlocal case. In Section 3, we present our new approach of inextensibility and discuss how we determine the rotation angle for the tangent vector from SBT. Section 4 is devoted to numerical methods and describes the tools we use for efficient evaluation of nonlocal SBT, namely special quadratures for nearly singular and singular integrals and Ewald splitting to accelerate sums in periodic systems. In Section 5, we show how our ``weak formulation of inextensibility'' gives improved accuracy over the ``strong formulation of inextensibility'' in \cite{ts04, ehssan17}. Section 6 shows how we can seamlessly introduce cross-linkers and gives some preliminary results, and Section 7 gives our conclusions. 
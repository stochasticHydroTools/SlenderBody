We first require a constitutive equation for the bending force density on the fibers. We use the Euler beam model, so that the bending force is given by
\begin{equation}
\label{eq:bforce}
    \V{f}^E=-E\V{X}_{ssss}\eqd \Lop{F}\V{X},
\end{equation}
where the linear operation $\Lop{F}\V{X}$ gives $ \V{f}^E$ with proper treatment of the ``free fiber'' boundary conditions in Eq.\ \eqref{eq:frBCs}. See Section \ref{sec:rsc} for details on how we construct a discrete form of $\Lop{F}$. 

It is easy to see that the boundary conditions in Eq.\ \eqref{eq:frBCs} cause the total force and torque on each fiber due to $\V{f}^E$ to be zero, 
\begin{gather}
\label{eq:totfe}
\int_{0}^{L} \V{f}^E \, ds = -E\V{X}_{sss}\Big \rvert^{L}_{0} = \V{0}, \quad \text{and}\\[2 pt]
\left(\int_{0}^{L} \V{f}^E \times \V{X} \, ds\right)^\ell = -E\int_{0}^{L} X^j X^k_{ssss} - X^k X^j_{ssss} \, ds = -E\int_{0}^{L} X^j_{ss}  X^k_{ss} -X^k_{ss} X^j _{ss} = 0. 
\end{gather}
Here superscripts denote vector components and $(j,k,\ell)$ is a cyclic permutation of $(1,2,3)$. In the torque equation, the free fiber boundary conditions lead to the cancellation of boundary terms that arise in integration by parts.

\subsection{Traditional formulation}
In the traditional formulation of inextensibility \cite{ts04}, Eq.\ \eqref{eq:inex} is differentiated with respect to time. Then, $s$ and $t$ derivatives are interchanged to yield
\begin{gather}
\label{eq:inexdt}
\left(\ddt{\Xs}\right)_s \cdot \Xs = 0. 
\end{gather}
In \cite{ts04}, the system was closed by substituting Eq.\ \eqref{eq:mobeqn} into Eq.\ \eqref{eq:inexdt}, where the total fiber force density is given by $\V{f}=\V{f}^E + \V{\lambda}$. By assuming that $\V{\lambda}=(T\Xs)_s$, where $T(s)$ is the scalar line tension, we obtain the \textit{line tension equation} of \cite{ts04}
\begin{equation}
\label{eq:lineT}
\left(\Lop{M}\left[\V{X}\right]\left(\Lop{F}\V{X}+(T\Xs)_s\right)\right)_s  \cdot \V{X}_s = 0. 
\end{equation}
While this equation is linear in $T$, it is highly nonlinear in $\V{X}$, since the operation $\Lop{F}\V{X}$ gives fourth derivatives of $\V{X}$. Even in the absence of any nonlocal interactions (i.e. if $\Lop{M}=\ML$) and zero background flow ($\V{U}_0=\V{0}$), the line tension equation still has terms of the form $\V{X}_{sss}\cdot \V{X}_{sss}$ (see \cite[~Eq. (13)]{ts04}). Because Eq.\ \eqref{eq:lineT} is a boundary value problem enforced pointwise along the fiber, we refer to it as a \textit{strong formulation of inextensibility}. 

\subsection{New formulation \label{sec:geo}}
In our approach, the variable of interest is $\Xs(s,t)$, rather than $\V{X}(s,t)$, which can be obtained by integration modulo a constant. Considering the evolution of $\Xs(s,t)$, Eq.\ \eqref{eq:inexdt} implies that 
\begin{equation}
\label{eq:omegadef}
\frac{\partial \Xs}{\partial t}(s,t) = \V{\Omega}(s, t) \times \Xs(s,t), 
\end{equation}
i.e. that the fiber evolution can be thought of as rotations of $\Xs$ on the unit sphere. Now, at each fiber point, we uniquely define an orthonormal coordinate system using Euler angles $\theta(s,t)$ and $\phi(s,t)$. We represent the unit tangent vector $\Xs(s,t)$ as
\begin{equation}
\Xs(s,t)=\Xs(\theta(s,t),\phi(s,t)) = \begin{pmatrix} \cos{\left(\theta(s,t)\right)} \cos{\left(\phi(s,t)\right)}\\[2 pt] \sin{\left(\theta(s,t)\right)} \cos{\left(\phi(s,t)\right)} \\[2 pt] \sin{\left(\phi(s,t)\right)} \end{pmatrix}.
\end{equation}
We define $\theta$ to be single-valued at $\phi=\pi/2$ by setting $\theta \left(\phi=\pm \pi/2\right)=0$. A choice of normal vectors that are always orthonormal to $\Xs$ on the unit sphere is
\begin{equation}
\label{eq:nangles}
\V{n}_1 =  \begin{pmatrix} -\sin{\theta}\\[2 pt] \cos{\theta}\\[2 pt]0 \end{pmatrix} \qquad \V{n}_2 =  \begin{pmatrix} -\cos{\theta} \sin{\phi}\\[2 pt] -\sin{\theta} \sin{\phi} \\[2 pt] \cos{\phi} \end{pmatrix}. 
\end{equation}
Other choices are also possible. Because $\V{n}_1$ and $\V{n}_2$ can be determined uniquely from $\Xs$, we refer to their dependencies from here forward as $\V{n}_j[\Xs(s,t)]$, for $j=1, 2$. Since $\theta$ is single-valued at $\phi=\pi/2$, each component of the orthonormal coordinate system $(\Xs,\V{n}_1,\V{n}_2)$ is smooth when $\Xs$ is smooth. 

Because $\Xs \times \Xs=\V{0}$, $\V{\Omega}(s,t)$ can be represented by linear combinations of $\V{n}_1$ and $\V{n_2}$. We let 
\begin{equation}
\label{eq:omdef}
\V{\Omega}(s,t) \eqd \V{\Omega}\left(\Xs(s,t),\V{g}(s,t)\right) \eqd g_1(s,t)\V{n}_2[\Xs(x,t)]-g_2(s,t)\V{n}_1[\Xs(s,t)],  
\end{equation}
where $g_1(s,t)$ and $g_2(s,t)$ are two specific unknown functions in $L^2:[0,L]$. Eq.\ \eqref{eq:omdef} implies that, by the right-handedness of the coordinate system $(\V{X}_s, \V{n}_1, \V{n}_2)$, 
\begin{equation}
\label{eq:Xsupdate}
\ddt{\Xs} = \V{\Omega}\left(\Xs(s,t),\V{g}(s,t)\right) \times \Xs(s,t) = g_1(s,t)\V{n}_1[\Xs(s,t)] + g_2(s,t)\V{n}_2[\Xs(s,t)]. 
\end{equation}
%The position of the fiber $\V{X}$ can be recovered by 
%\begin{equation}
%\label{eq:XfromXs}
%\V{X}(s)= \V{X}(0)+\int_0^s \V{X}_s(\theta(s),\phi(s)) \, ds, 
%\end{equation}
The velocity of any given fiber centerline can now be written as 
\begin{equation}
\label{eq:velfib}
\ddt{\V{X}}(s,t) = \V{U}(t)+\int_0^s  \sum_{j=1}^2 g_j(s',t)\V{n}_j[\Xs(s',t)] \,ds' . 
\end{equation}
where $\V{U}(t)$ is a rigid body translation. 

%Here we have defined a linear operator $\bm{K}$ acting on a vector $\bm{g}=(g_1,g_2,\bm{U})$. 
%Substituting Eq.\ \eqref{eq:defk} into Eq.\ \eqref{eq:fibevcont}, we have 
%\begin{equation}
%\label{eq:svcont}
%\left(\bm{K}\bm{g}\right)(s) = \bm{M}\left(\bm{f}^E+\bm{\lambda}\right)+\bm{U}_0(\bm{X}). 
%\end{equation}
%Therefore, can be viewed as a rotational velocity for $\V{X}_s$ on the unit sphere. The notation here indicates that $\V{\alpha}$ determines $g_1$ and $g_2$ and therefore $\V{\Omega}$. We will use this to update the configuration in our temporal integration schemes by rotating the tangent vector $\V{X}_s$ and then computing $\V{X}$ using \eqref{eq:XfromXs}; this strictly enforces inextensibility.
%Because our formulation involves integral rather than differential equations, we refer to it as a weak formulation of inextensibility. In fact, taking the derivative of Eq.\ \eqref{eq:svcont} with respect to $s$ and taking the inner product with $\bm{X}_s$ gives the line tension equation of \cite{ts04}. 

\subsection{Principle of virtual work}
So far, we have only laid out the new geometry we use to evolve the filament. In fact, the formulation of Section \ref{sec:geo} can still be coupled to a line tension equation to obtain the fiber velocity. We choose to close our formulation differently, in the process eliminating the need for an auxiliary line tension equation. 

The principle of virtual work states that the constraint forces $\V{\lambda}(s,t)$ do no work for any choice of $\V{g}(s,t)$ \cite{varibook}. Because this constraint holds for all $t$, we drop for the moment the explicit dependence on $t$ in the notation. To impose this constraint, we use the $L^2$ inner product to compute the total dissipated power on the fluid from $\V{\lambda}$. 
\begin{align}
\label{eq:15}
\mathcal{P}& =\Bigg{\langle} \V{\lambda},\ddt{\V{X}} \Bigg{\rangle} = \int_0^L \left(\V{U}+\int_0^{s'} g_1(s) \V{n}_1[\Xs(s)] + g_2(s)\V{n}_2[\Xs(s)] \,ds\right) \cdot \V{\lambda}\left(s'\right) \, ds'\\[2 pt] 
\label{eq:16}
& = \V{U} \cdot \int_0^L \V{\lambda}\left(s'\right) \, ds' +\int _0^L ds \int_{s}^L \left(g_1(s)\V{n}_1[\Xs(s)] + g_2(s) \V{n}_2[\Xs(s)] \right) \cdot \V{\lambda}(s') \, ds'\\[2 pt]
\label{eq:Kstarcont}
& = \V{U} \cdot \int_0^L \V{\lambda}\left(s'\right) \, ds' + \int_0^L \left(g_1(s)\V{n}_1[\Xs(s)] + g_2(s) \V{n}_2[\Xs(s)] \right) \cdot  \left(\int_{s}^L  \V{\lambda}(s') \, ds'\right) \, ds=0 
\end{align}
Here Eq.\ \eqref{eq:16} is obtained from Eq.\ \eqref{eq:15} via a simple change of integration variables. Since Eq.\ \eqref{eq:Kstarcont} must hold for all $\bm{U}$ and all sufficiently smooth $g_1$ and $g_2$, we must have that, for all $s$, 
\begin{equation}
\label{eq:noworkcont}
\begin{pmatrix} \left(\int_s^L \V{\lambda}(s')\, ds'\right) \cdot \V{n}_1[\Xs(s)]\\[2 pt] \left(\int_s^L \V{\lambda}(s')\, ds'\right) \cdot\V{n}_2[\Xs(s)]\\[2 pt] \int_0^L \V{\lambda}(s') ds' \end{pmatrix} = \begin{pmatrix} 0 \\[2 pt] 0\\[2 pt] \bm{0}\end{pmatrix}. 
\end{equation}
This constraint can be used to obtain a closed form for $\V{\lambda}$. The first and second components of Eq.\ \eqref{eq:noworkcont} taken together tell us that $\left(\int_{s}^L  \V{\lambda}\left(s'\right) \, ds'\right)$ is orthogonal to both normal vectors. Therefore, $\left(\int_{s}^L  \V{\lambda}\left(s'\right) \, ds'\right)$ is in the direction of $\V{X}_s(s)$ and can be written as
\begin{equation}
\label{eq:tsalm}
\int_{s}^L  \bm{\lambda}\left(s'\right) \, ds' = -T(s)\Xs,
\end{equation}
for some scalar function $T(s)$ with $T(s=L)=0$. Differentiating both sides of Eq.\ \eqref{eq:tsalm} with respect to $s$, we obtain 
\begin{equation}
\label{eq:lamval}
\V{\lambda}(s) = \left(T(s)\Xs\right)_s, 
\end{equation}
which is the form of Tornberg and Shelley \cite{ts04}. Thus our derivation shows that the form of $\V{\lambda}$ taken in \cite{ts04} is equivalent to the principle that the constraint forces perform no virtual work \cite{varibook}. 

Now, returning to the third constraint in Eq.\ \eqref{eq:noworkcont}, $\int_0^L \V{\lambda}(s) \, ds=\V{0}$, and substituting Eq.\ \eqref{eq:lamval} for $\V{\lambda}$, we obtain 
\begin{equation}
\label{eq:BCT}
T(L)\Xs(L) - T(0)\Xs(0)=\V{0}. 
\end{equation}
Since $T(L)=0$, Eq.\ \eqref{eq:BCT} implies that $T(0)=0$ as well, since neither of the tangent vectors is identically 0. So we obtain $T(0)=T(L)=0$, which is exactly the boundary condition for the line tension equation in \cite{ts04}. 

In this sense, the constraint equations \eqref{eq:15}$-$\eqref{eq:noworkcont} are all equivalent to the line tension equation used in prior work \cite{ts04}. Because we showed the equivalence by enforcing Eq.\ \eqref{eq:Kstarcont} ``for every choice of $g_1(s)$ and $g_2(s)$,'' we refer to our new formulation as a \textit{weak formulation of inextensibility}. In the next section, we show how choosing a suitable basis for $g_1(s)$ and $g_2(s)$  leads to a method with a Galerkin-like feel.  

\subsection{$L^2$ weak formulation}
\label{sec:numinex}
In this section, we introduce an $L^2$ weak formulation that will later lead to a novel numerical method.

The key idea is to expand the unknown functions $g_1(s,t)$ and $g_2(s,t)$ as, 
\begin{equation}
\label{eq:basis}
g_j(s,t) = \sum_k \alpha_{jk}(t) T_k(s), \quad \text{for $j=1, 2$}, 
\end{equation}
where $T_k(s)$ are sufficiently smooth scalar-valued basis functions for $L^2:[0,L]$. Here we have re-introduced the dependence on time $t$ to show that the basis functions are constant in time while their coefficients vary. 

Substituting the expansion of Eq.\ \eqref{eq:basis} into Eq.\ \eqref{eq:velfib}, we obtain
\begin{equation}
\label{eq:du}
\ddt{\V{X}}(s,t) =\bm{U}(t) + \int_0^s \sum_{j=1}^2\sum_k \alpha_{jk}(t) T_k\left(s'\right) \V{n}_j[\Xs(s',t)] \, ds' \eqd (\Lop{K}\left[\V{X}(\cdot,t)\right]\V{\alpha}(t))(s),
\end{equation}
where the linear operator $\Lop{K}\left[\V{X}(\cdot,t)\right]$ acts on $\V{\alpha}(t)=(\alpha_{jk}(t),\bm{U}(t))$ to give an inextensible velocity on the filament centerline ($\V{\alpha}$ parameterizes inextensible fiber motions). Note the dependence of $\Lop{K}$ on $\V{X}$ since $\Lop{K}$ involves the normal vectors $\V{n}_1$ and $\V{n}_2$. Using Eq.\ \eqref{eq:du}, The dynamical equation, \eqref{eq:mobeqn}, now becomes
\begin{equation}
\label{eq:veleqn}
(\Lop{K}\left[\V{X}(\cdot,t)\right]\V{\alpha}(t))(s) = \V{U}_0\left(\V{X}(s,t),t\right) + \left(\Lop{M}\left[\V{X}(\cdot,t)\right]\left(\V{f}^E(\cdot,t) +\V{\lambda}(\cdot,t)\right)\right)(s)
\end{equation}

This new dynamical equation, \eqref{eq:veleqn} is supplemented by enforcing Eq.\ \eqref{eq:noworkcont} in an $L^2$ weak sense. We begin by substituting the representation of $g_j$ in Eq.\ \eqref{eq:basis} into the power equation, \eqref{eq:15}, to obtain (again dropping the time dependence in the notation)
\begin{align}
\label{eq:26}
\mathcal{P} & = \bigg{\langle} \V{\lambda}(\cdot),\Lop{K}[\V{X}(\cdot)]\V{\alpha}\bigg{\rangle}= 0 \\[2 pt]
\label{eq:Kstardef}
\mathcal{P} & = \V{U} \cdot \int_0^L \V{\lambda}(s) \, ds + \int_0^L \left(\int_0^{s} \sum_{j=1}^2\sum_{k} \alpha_{jk} T_k(s') \V{n}_j[\Xs(s')] \, ds'\right)\cdot \V{\lambda}\left(s\right) \, ds\\[2 pt]
\nonumber & \eqd \bigg{\langle} \Lop{K}^*[\V{X}(\cdot)]\V{\lambda}(\cdot), \V{\alpha}\bigg{\rangle} =0. 
\end{align}
Eq.\ \eqref{eq:Kstardef} formally defines the $L^2$ adjoint of $\Lop{K}$ as $\Lop{K}^*$. Since the power from the constraint forces must be zero for any inextensible motion (any $\V{\alpha}$), we define $\Lop{K}^*$ by setting each term of Eq. \eqref{eq:Kstardef} to zero. That is,
\begin{equation}
\label{eq:noworkcontL2}
\Lop{K}^*[\V{X}(\cdot)]\V{\lambda}(\cdot):=\begin{pmatrix} \int_0^L \left(\int_0^{s} T_k(s') \V{n}_1[\Xs(s')] \, ds'\right)\cdot \bm{\lambda}(s) \, ds\\[2 pt] \int_0^L \left(\int_0^{s} T_k(s') \V{n}_2[\Xs(s')] \, ds'\right)\cdot \bm{\lambda}(s) \, ds\\[2 pt] \int_0^L \bm{\lambda}(s) \, ds \end{pmatrix} = \begin{pmatrix} 0 \\[2 pt] 0\\[2 pt] \V{0}\end{pmatrix}, 
\end{equation}
where the first two vector components hold for all $k$ and the last vector component holds in each of the three directions.

A saddle-point system can now be written for $\bm{\lambda}$ and $\bm{\alpha}$ by combining Eqs.\ \eqref{eq:veleqn} and \eqref{eq:noworkcontL2}. Dropping all dependencies on $s$ and $t$ and setting $\V{f}^E=\Lop{F}\V{X}$, we have
\begin{equation}
\label{eq:saddleL2}
    \begin{pmatrix}
    -\Lop{M}[\V{X}] & \Lop{K}[\V{X}]\\[4 pt]
    \Lop{K}^*[\V{X}] & \bm{0}
    \end{pmatrix}
    \begin{pmatrix} 
    \V{\lambda}\\[4 pt]
    \V{\alpha}
    \end{pmatrix} =  \begin{pmatrix} 
    \Lop{M}[\V{X}]\left(\Lop{F}\V{X}\right)+\V{U}_0\left(\V{X}\right)\\[4 pt]
    \bm{0}\end{pmatrix}.
\end{equation}

In continuum, solving this system for $\V{\lambda}$ and $\V{\alpha}$ is equivalent to solving the line tension equation for $T$ and letting $\V{\lambda}=(T\V{X}_s)_s$. Numerically, Eq.\ \eqref{eq:saddleL2} allows us to avoid solving the problematic line tension equation. \cmt{Summary of eqns?}

\iffalse
Now, we can eliminate $\bm{\lambda}$ by a Schur complement approach. Applying the formal (psuedo)inverse of $\bm{M}$ to both sides of Eq.\ \eqref{eq:veleq} and then applying $\bm{K}^*$ to both sides, we obtain
\begin{equation}
\left(\left(\bm{K}^*\bm{M}^{-1} \bm{K}\right)\bm{\alpha}\right)(s) = \bm{K}^*\left(\bm{f}^E+\bm{M}^{-1}\bm{U}_0\right)=\bm{K}^*\left(\bm{L}\bm{X}+\bm{M}^{-1}\bm{U}_0\right),
\end{equation}
where we have used Eq.\ \eqref{eq:noworkcontL2} to eliminate $\bm{\lambda}$. If we now apply the (pseudo)inverse of the operator $\bm{K}^*\bm{M}^{-1}\bm{K}$, then apply $\bm{K}$ on the left to both sides, we obtain a solution for the velocity of the fiber centerline of the form
\begin{equation}
\frac{\partial \bm{X}}{\partial t}(s) = \left(\bm{K}\bm{\alpha}\right)(s) = \left(\bm{K}\left(\bm{K}^*\bm{M}^{-1}\bm{K}\right)^{-1}\bm{K}^*\right)\left(\bm{L}\bm{X}+\bm{M}^{-1}\bm{U}_0\right):
=\bm{N}\left(\bm{L}\bm{X}+\bm{M}^{-1}\bm{U}_0\right). 
\end{equation}
\fi
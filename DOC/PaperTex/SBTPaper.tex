\documentclass[12pt]{article}
\usepackage{lipsum}
\linespread{1.5} 
\usepackage[utf8]{inputenc}
\usepackage[utf8]{inputenc}
\usepackage{graphicx,epstopdf,subfigure,mathtools,mathrsfs} 
\usepackage[normalem]{ulem}
\PassOptionsToPackage{usenames,dvipsnames}{xcolor}
\usepackage[usenames,dvipsnames]{xcolor}
\newcommand{\cmt}[1]{\color{blue}#1\normalcolor}
% Path for Figures
\graphicspath{{../}}

% VECTOR AND MATRIX NOTATION
\newcommand{\V}[1]{\boldsymbol{#1}}                 % vector notation
\newcommand{\M}[1]{\boldsymbol{#1}}
\newcommand{\Lop}[1]{\mathcal{#1}}
\newcommand{\bm}[1]{\boldsymbol{#1}}
\newcommand{\ind}[2]{{#1}_{#2}}
\newcommand{\ddt}[1]{\frac{\partial #1}{\partial t}}


\global\long\def\D#1{\Delta#1} 
\global\long\def\d#1{\delta#1} 

\global\long\def\norm#1{\left\Vert #1\right\Vert }
\global\long\def\abs#1{\left|#1\right|}
\global\long\def\bm#1{\V#1}
\global\long\def\eqd{:=}

\global\long\def\grad{\M{\nabla}}
%\global\long\def\av#1{\left\langle #1\right\rangle }

% NOTATION SPECIFIC TO THIS PAPER
\global\long\def\intmat{\mathcal{D}^{-1}}
\global\long\def\Slet#1{S\left(#1\right)}
\global\long\def\Dlet#1{D\left(#1\right)}
\global\long\def\Knel#1{K\left(#1\right)}
\global\long\def\Xs{\V{X}_s}
\global\long\def\EPMI{\frac{1}{8\pi\mu}}
\global\long\def\ML{ \M{M}^\text{L}}
\global\long\def\MFP{\Lop{M}^\text{FP}}
\global\long\def\MJF{\Lop{M}^\text{JF}}
\global\long\def\aRPY{a^*}%\displaystyle{\epsilon L \sqrt{\frac{3}{2}}}}

\global\long\def\eqd{:=}

\usepackage{hyperref}
\hypersetup{
    colorlinks=false,
    pdfborder={0 0 0},
}

% Set the margins for 1" around
\setlength{\topmargin}{-0.5in}
\setlength{\oddsidemargin}{0.1in}
\setlength{\evensidemargin}{0.0in}
\setlength{\textwidth}{6.4in}
\setlength{\textheight}{8.9in}

\title{An integral-based spectral method for slender body hydrodynamics}
\author{Ondrej Maxian and Aleksandar Donev}

\begin{document}

\maketitle

\abstract{\cmt{To be added later.}%Actin filaments, microtubules, and other inextensible filaments play a vital role in the cell division process. The multiscale nature of these objects (aspect ratios as large as 10,000) makes their numerical simulation challenging. We describe a new method for rapidly computing the dynamic behavior of inextensible slender biological filaments in Stokes flow. Our approach is based on using an integral equation to restrict the velocity obtained from slender body theory to the space of inextensible motions. By using a spectral discretization of the local and nonlocal slender body theory operators and a second-order temporal discretization, we achieve improved robustness and accuracy, both of which are demonstrated here through numerical examples. We conclude by applying our method to a cross-linked actin mesh. 
}

\section{Introduction}
This paper is concerned with the interactions of long, thin, inextensible filaments with a viscous fluid. Examples of these interactions abound in biology, engineering, physics, and medicine. In biology, the swimming mechanisms of flagellated organisms have been of interest for decades, with initial focus on using force and torque balances to understand swimming mechanics \cite{chwang1971note, berg1973bacteria, brennen1977fluid, lauga2009hydrodynamics}, and more recent studies on flagellar bundling \cite{lim2012fluid, maier2016magnetic}. In physics and engineering, suspensions of high-aspect-ratio fibers have been observed to display non-Newtonian, viscoelastic behavior both experimentally \cite{fibexps} and computationally \cite{mackfibs, kochsumfig}. 

Our particular area of interest is the simulation of filaments that make up the cell cytoskeleton. These filaments, which include microtubules and actin filaments and have aspect ratios from 100 to 10,000, maintain the cell's structure and control the mechanics of the cell division process \cite{alberts}. In vivo, actin filaments are generally bound together into networks by cross-linking proteins, the properties of which determine the viscoelastic behavior of the cytoskeleton \cite{wagner2006cytoskeletal, head2003deformation, ahmed2015dynamic}. While much of the recent work on slender body hydrodynamics has been in the context of microtubules and the positioning of the cellular spindle \cite{ehssan17, shelley2016dynamics}, to our knowledge there has been no systematic study of the influence of hydrodynamic solvent interactions on the mechanics of cross-linked actin networks. One of the goals of this paper is therefore to develop an efficient numerical technique for the simulation of $\mathcal{O}(100-1000)$ cross-linked actin networks, taking into account their interactions with a viscous (zero Reynolds number) solvent. 

Prior to the year 2000, tools for analytical analysis and numerical simulation of filaments in Stokes flow were generally developed in parallel, but separately. For slender filaments, a useful approach for both analysis and computation is to reduce the problem from three dimensions to one by assuming a certain distribution of singularities along the filament centerline. This approach, referred to as ``slender body theory'' (SBT), was first introduced by Hancock \cite{hancock1953self} and later expanded upon by Batchelor \cite{batchelor1970slender}. By using the method of matched asymptotics, Keller and Rubinow were the first to derive an SBT that is uniformly accurate in the fiber slenderness ratio $\epsilon = $ radius/length. Johnson further developed the theory by adding higher order corrections and correctly treating a fiber with free ends \cite{johnson}, and G\"otz re-derived the SBT of Keller and Rubinow in a more general context, allowing him to apply the theory to Oseen's and Poisson's equations \cite{gotz2001interactions}. 

Because the SBTs of Keller and Rubinow, Johnson, and G\"otz are uniformly accurate in powers of $\epsilon$, they have formed the basis of most modern analysis. To this end, Mori \textit{et al.} recently showed that these singularity solutions solve a well-posed Stokes problem with non-standard boundary conditions on the filament surface \cite{mori2018theoretical, morifree}. Koens and Lauga also showed that the SBT singularity solution can be recovered by matched asymptotic expansion of the full surface boundary integral formulation of Stokes flow \cite{koens2018boundary}. 

On the numerical side, non-SBT based techniques for the simulation of fibers in Stokes flow have been in use for many decades. The most prevelant among these are so-called immersed boundary methods, in which the fibers are discretized by a series of marker points, each of which is assigned a force according to the fiber physics. In one approach, the force is regularized by spreading to a background fluid grid using a compactly supported regularized delta function. The fluid velocity is then found by solving the fluid equations on the grid, and interpolation of this velocity field back onto the marker points yields the marker velocities (this is the IB method of Peskin and collaborators \cite{peskin1972flow, peskin2002acta}). In a similar yet distinct approach, the force can also be regularized using some Gaussian-like blob or cutoff function, and the Stokes equations can be solved analytically for that forcing to obtain the marker velocities (the method of regularized Stokeslets of Cortez and collaborators \cite{cortez2001method, cortez2005method}). 

In both of these methods, proper modeling of slender fibers requires that the width of the regularization function be on the order of the fiber radius \cite{ttbring08}. Since the regularization width is also tied to the fiber discretization spacing (and, in the IB method, the fluid grid spacing), slender fibers must be discretized with many more points than would be necessary in a continuum, SBT-based approach. While this limitation can be partially overcome with adaptive mesh refinement \cite{griffith2007adaptive}, grid coarsening with local velocity correction \cite{maxian19}, and kernel independent fast multipole methods (to accelerate many-body sums) \cite{rostami2016kernel}, the fact remains that to achieve controlled accuracy for dilute suspensions of many fibers, IB methods are generally more expensive than continuum-based approaches. 

Recognizing this, Shelley and Ueda were the first to apply a continuum-based SBT approach to the immersed slender fiber problem. By designing a numerical method around the analytical results of slender body theory, they reduced the complexity of the numerical computations from three dimensions to one \cite{shelley1996nonlocal, shelley2000stokesian}. Their formulation, however, relies on the filament being a closed loop, thus excluding many problems from biology, engineering, and physics where the filament ends are free. 

Tornberg and Shelley treated inextensible filaments with free ends using an SBT-based numerical method \cite{ts04}. In their approach, inextensibility is preserved by deriving an auxiliary integro-differential equation for the line tension in the filament, which acts as a Lagrange multiplier to preserve inextensibility. The line tension PDE combined with the SBT evolution equation for the fiber position gives a closed system for the filament dynamics. The method of \cite{ts04} has since been used in applications with flexible (and sometimes fluctuating) filaments  \cite{manikantan2013subdiffusive, young2009hydrodynamic}, and was also extended to simulate falling rigid fibers, the novelty there being that many of the SBT-related integrals can be done analytically \cite{tornberg2006numerical}. More recently, Nazockdast \textit{et al.} modified the approach of Tornberg and Shelley to make it feasible to simulate many-body cellular fiber assemblies. By replacing the second-order spatial discretization of Tornberg and Shelley with a spectral spatial discretization and utilizing a kernel-independent FMM to accelerate sums, they developed a parallel algorithm that makes it possible to simulate $\mathcal{O}(1000)$ fibers in linear time \cite{ehssan17}. 

Despite these recent advances, a need still exists for a re-examination of SBT-based approaches for inextensible fibers. For example, the line tension equation of \cite{ts04} involves multiplications of high-order (as high as four) derivatives of the fiber position function. Thus in the spectral formulation of \cite{ehssan17}, aliasing results, and a loss of accuracy occurs in the spatial discretization. In addition, the ``inextensibility'' of the filaments is still subject to discretization error and requires inserting a penalty term into the line tension equation that reduces the discrete extensibility \cite{ts04}. For fibers tugged by cross-linkers, this penalty parameter will be large, introducing artificial stiffness into the problem. And finally, the most recent contribution of \cite{ehssan17} neglects higher-order corrections in the SBT equations when filaments are close together, leading to a loss of accuracy when filaments approach each other. 

The primary focus of this paper is on a new formulation for inextensible filaments. In our approach, the fibers are evolved via a rotation of the tangent vector on the unit sphere, and the new positions are then obtained by integration. In this way, we maintain exact inextensibility of the fibers without introducing a penalty parameter. We couple this advance with the latest techniques \cite{barLud, tornquad} for efficient evaluation of each component in nonlocal slender body theory to present a method with improved accuracy and robustness over that of \cite{ehssan17}. 

The rest of this paper is laid out as follows: we begin in Section 2 by introducing the necessary SBT equations for both the local and nonlocal case. In Section 3, we present our new approach of inextensibility and discuss how we determine the rotation angle for the tangent vector from SBT. Section 4 is devoted to numerical methods and describes the tools we use for efficient evaluation of nonlocal SBT, namely special quadratures for nearly singular and singular integrals and Ewald splitting to accelerate sums in periodic systems. In Section 5, we show how our ``weak formulation of inextensibility'' gives improved accuracy over the ``strong formulation of inextensibility'' in \cite{ts04, ehssan17}. Section 6 shows how we can seamlessly introduce cross-linkers and gives some preliminary results, and Section 7 gives our conclusions. 

\section{Slender body theory}
For completeness, we begin by summarizing the slender body theories of \cite{krub, johnson, gotz2001interactions}, here following in particular Johnson \cite{johnson} and G\"otz \cite{gotz2001interactions}. 

\subsection{Single filament}
Introducing notation first, let $\V{X}(s)$ be the centerline of a filament, parameterized by arclength on $s \in [0,L]$, where $L$ is the fiber length. The tangent vector is $\Xs(s)=\displaystyle{\frac{\partial \V{X}}{\partial s}}$ and has unit length. The fiber has physical radius $a(s) = r \rho(s)$, where $0 \leq \rho(s) \leq 1$, and slenderness ratio $\epsilon = r/L$. Let the force per unit length on the fiber centerline be denoted as $\V{f}(s)$. We denote the background flow at an arbitrary point in the fluid as $\V{U}_0(\V{x},t)$. In this section we study a specific instant in time, so we refer to $\V{U}_0$ as $\V{U}_0(\V{x})$. 

We first define the Stokeslet and doublet centered at $\V{x}_0$, 
\begin{equation}
\label{eq:Slet}
\Slet{\V{x},\V{x}_0} = \frac{\M{I}+\hat{\V{R}}\hat{\V{R}}}{\norm{\V{R}}} \qquad \text{ and } \qquad 
\Dlet{\V{x},\V{x}_0} = \frac{\M{I}-3\hat{\V{R}}\hat{\V{R}}}{\norm{\V{R}}^3},
\end{equation}
where $\V{R} = \V{x}-\V{x}_0$ and $\hat{\V{R}}=\V{R}/\norm{\V{R}}$. The Stokeslet is the fundamental solution to the Stokes equations for a delta-function forcing at $\V{x}_0$, while the doublet (Laplacian of the Stokeslet) is the fundamental solution for a mass source dipole at $\V{x}_0$. 

The idea of SBT is to introduce an ansatz for the flow field away from the fiber centerline of the form
\begin{align}
\label{eq:sbtsd}
\V{u}(\V{x}) - \V{U}_0(\V{x}) = & \EPMI \int_0^L \left(\Slet{\V{x},\V{X}(s)}+\beta(s)\Dlet{\V{x},\V{X}(s)}\right)\V{f}(s) \, ds\\[2 pt] 
\label{eq:Kdef}
\eqd & \EPMI \int_0^L \Knel{\V{x},\V{X}(s), \beta(s)}\V{f}(s) \, ds. 
\end{align}
In Eq.\ \eqref{eq:Kdef}, we have defined an integral kernel $K$ that is a combination of a Stokeslet and a doublet with strenght $\beta$. Using the method of matched asymptotic expansions, the integral in Eq.\ \eqref{eq:sbtsd} can be computed analytically on the surface of the fiber to $\mathcal{O}(\epsilon)$ (see \cite{gotz2001interactions, koens2018boundary} for details on these integrals). The value of $\beta$ comes from imposing the boundary condition that the velocity on the fiber surface be constant to $\mathcal{O}(\epsilon)$ (Mori \textit{et al.} \cite{mori2018theoretical, morifree} refer to this as the ``fiber integrity condition''). This yields the solution for the velocity in the fluid as 
\begin{equation}
\label{eq:sbt2}
\V{u}(\V{x}) - \V{U}_0(\V{x}) =\EPMI \int_0^L  \Knel{\V{x},\V{X}(s), \frac{a(s)^2}{2}}\V{f}(s) \, ds. 
\end{equation}
Eq.\ \eqref{eq:sbt2} is not defined on the centerline of the fiber. Physically, however, the velocity of the fiber centerline $\ddt{\V{X}}(s)$ should be equal to \textit{the average of Eq.\ \eqref{eq:sbt2} around a cross-section of the fiber}. This averaging can be done asymptotically in $\epsilon$ \cite{gotz2001interactions} to obtain 
\begin{gather}
\label{eq:onefib}
\ddt{\V{X}}(s)-\V{U}_0\left(\V{X}(s)\right)= \ML[\V{X}(s)]\V{f}(s) + \left(\MFP\left[\V{X}(\cdot)\right]\V{f}(\cdot)\right)(s), \text{ where} \\[2 pt]
\label{eq:ML}
\ML[\V{X}(s)]= \EPMI \left(c(s)(\V{I}+\Xs(s)\Xs(s)) +  (\V{I}-3\Xs(s)\Xs(s))\right),\\[2 pt]
\label{eq:Mfp}
\left(\MFP\left[\V{X}(\cdot)\right]\V{f}(\cdot)\right)(s) =  \EPMI \int_{0}^L \Slet{\V{X}(s),\V{X}(s')} \V{f}(s') -\left(\frac{\V{I}+\Xs(s)\Xs(s)}{|s-s'|}\right) \V{f}(s) \, ds'. 
\end{gather}
Here $\ML[\V{X}(s)]$ is the local drag matrix which gives the velocity contribution from forcing at points $\mathcal{O}(\epsilon)$ away from $\V{X}(s)$.  The linear operator $\MFP$ gives the action of the so-called finite part integral. The first term in the integrand is the Stokeslet, and the second term is the ``common'' part in the matched asymptotic expansion that comes from expansion of the Stokeslet around $s'=s$. Physically, the finite part integral gives the velocity contribution from forcing at points $\mathcal{O}(1)$ away from $\V{X}(s)$. Thus while both terms in the integrand are singular, their difference is finite \cite{ts04}. 

In Eq.\ \eqref{eq:ML}, the leading order local drag coefficient is given by \cite{gotz2001interactions}
\begin{equation}
\label{eq:unmodc}
c(s) = \text{log}\left(\frac{4s(L-s)}{a(s)^2}\right)
\end{equation}
and is singular without proper decay of $a(s)$ at $s=0$ and $s=L$. Johnson \cite{johnson} was the first to show that, when $a(s)$ decays near the fiber endpoints as $2\epsilon\sqrt{s(L-s)}$ (i.e. ellipsoidally), Eq.\ \eqref{eq:onefib} gives a uniformly accurate (to $\mathcal{O}(\epsilon^2 \text{log}\, \epsilon))$ approximation to the Stokeslet strength for a given velocity on the fiber boundary. The accuracy of $\mathcal{O}(\epsilon^2 \text{log}\, \epsilon)$ comes from his consideration of higher order singularities that make the velocity on the fiber cross section constant to $\mathcal{O}(\epsilon^2)$. That said, the nonlocal aspects of slender body theory are greatly simplified if we assume the filaments to be cylinders with spherical caps, so that $a(s) = r = \epsilon L$ on $s \in [0,L]$. In this case, Eq.\ \eqref{eq:unmodc} becomes singular at the filament ends, and we replace it with the local drag coefficient of \cite{morifree}, 
\begin{equation}
\label{eq:creg}
c(s) = \text{log}\left(\frac{2s(L-s)+2\sqrt{\epsilon^2 L^4 + s^2(L-s)^2}}{\epsilon^2 L^2}\right),
\end{equation}
which regularizes the leading order coefficient near the fiber ends at the cost of reduced asymptotc accuracy there. %For a straight fiber with constant forcing, our internal testing showed Eq.\ \eqref{eq:onefib} with leading order coefficient given by Eq.\ \eqref{eq:creg} to give an $\mathcal{O}(\epsilon^2)$ approximation to averaging Eq.\ \eqref{eq:sbt2} around a fiber cross section, with proportionality constants approaching infinity at the fiber endpoints (separate treatment of the endpoints shows Eq.\ \eqref{eq:onefib} to be an $\mathcal{O}(1)$ approximation to actual averaging at $s=0,L$). Johnson \cite{johnson} discusses how to treat the ends in a way that maintains $\mathcal{O}(\epsilon^2)$ accuracy everywhere; we do not concern ourselves with this here, for simplicity, and accept an $\mathcal{O}(1)$ error at the fiber endpoints. 

\subsection{Multiple filaments}
It remains to include in Eq.\ \eqref{eq:onefib} the perturbed flow due to other filaments, i.e. to account for hydrodynamic interactions between fibers. The simplest approach for this, which was taken by Tornberg and Shelley \cite{ts04}, is to simply evaluate Eq.\ \eqref{eq:sbt2} on the centerline of the other fibers. Nazockdast \textit{et al.} also adopted this, except they dropped the doublet term completely and included only the Stokeslet term. We take a different approach: inspired by the single fiber solution, we define the velocity induced by fiber $j$ on the centerline of fiber $i$ to be the average of Eq.\ \eqref{eq:sbt2} taken over a circular ring cross section of fiber $i$. For a single fiber ($i=j$), this definition gives Eq.\ \eqref{eq:onefib} to $\mathcal{O}(\epsilon^2 \text{log}\, \epsilon)$ (away from the filament ends) \cite{johnson}. For other filaments, because Eq.\ \eqref{eq:sbt2} is only valid to $\mathcal{O}(\epsilon)$, the velocity we obtain on other fibers is only correct to $\mathcal{O}(\epsilon)$. 

It is actually not difficult to derive a formula for the average of Eq.\ \eqref{eq:sbt2} around any fiber $i$ that does not intersect fiber $j$ if we make two observations. First, we recall that the flow induced by a sphere of radius $b$ centered at $\V{X}$ with forcing $\V{F}$ is given by 
\begin{equation}
\label{eq:sphvel}
\V{u}(\V{x}) - \V{U}_0(\V{x}) = \EPMI \Knel{\V{x},\V{X},\frac{b^2}{3}}\V{F}. 
\end{equation}
We therefore recognize that Eq.\ \eqref{eq:sbt2} is the flow field due to a line of spheres of radius $\displaystyle{b(s) = \sqrt{\frac{3}{2}}a(s)}$. Since we are considering a constant $a(s)=\epsilon L$, the velocity induced by the filament is equivalent to that due to a line of spheres of radius 
\begin{equation}
\label{eq:rpyradius}
a^*=\epsilon L \sqrt{\frac{3}{2}}.
\end{equation}

Our second observation is that the average of any smooth velocity field is the same to $\mathcal{O}(\epsilon^2)$, regardless of whether it is taken over a sphere of radius $\aRPY$ or a circular ring of radius $\epsilon L$ (as long as the sphere and ring have the same center). We can therefore replace our averaging over a circular ring cross section with averaging over a sphere of radius $\aRPY$. 

Combining these two observations, the non-local velocity field reduces to an average over a sphere of the flow induced by a collection of other spheres, all with radius $\aRPY$. Now, the flow created by a sphere of radius $a^*$ on another sphere of radius $a^*$ is given by the Rotne-Prager tensor \cite{rpyOG, PSRPY}, which for non-overlapping spheres is
\begin{equation}
\M{M}_\text{RPY}\left[\V{X},\V{Y}\right] = \EPMI \Knel{\V{X},\V{Y},\frac{2{\left(a^*\right)}^2}{3}}. 
\end{equation}
Substituting the equivalent sphere radius for a fiber, $\aRPY$ from Eq.\ \eqref{eq:rpyradius}, we obtain the kernel $\Knel{\ind{\V{X}}{i},\ind{\V{X}}{j},(\epsilon L)^2}$ for the contribution of a point $\ind{\V{X}}{j}$ on fiber $j$ to the velocity at point $\ind{\V{X}}{i}$ on fiber $i$. Integrating over the centerline of fiber $j$, we obtain a formula for the velocity at $s$ on fiber $i$ due to fiber $j$, which we define via
\begin{equation}
\label{eq:sbtother}
\MJF\left[\V{X}_i(s), \V{X}_j(\cdot)\right]\ind{\V{f}}{j}(\cdot) \eqd \EPMI \int_0^{L_j} \Knel{\ind{\V{X}}{i}(s),\ind{\V{X}}{j}(s'),(\epsilon L)^2}\ind{\V{f}}{j}(s') \, ds'. 
\end{equation}
Here we have again defined an operator $\MJF$ which acts linearly on $\ind{\V{f}}{j}$ to give the velocity on the centerline of filament $i$ solely due to filament $j$. 

Summing over filaments $j \neq i$ and adding to the terms from local drag, we have the total slender body velocity on filament $i$ given by
\begin{equation}
\label{eq:fibevcont}
\ddt{\ind{\V{X}}{i}}(s) -\V{U}_0(\ind{\V{X}}{i}(s))=  \ML [\ind{\V{X}}{i}(s)]\ind{\V{f}}{i}(s) + \left(\MFP\left[\ind{\V{X}}{i}(\cdot)\right]\ind{\V{f}}{i}(\cdot)\right)(s) + \sum_{j \neq i}\MJF\left[\V{X}_i(s), \V{X}_j(\cdot)\right]\ind{\V{f}}{j}(\cdot)
\end{equation}

In sum, given positions $\V{X}=\{\ind{\V{X}}{i}\}$ and force densities $\V{f}=\{\ind{\V{f}}{i}\}$ on the fiber centerlines, there exists a total mobility operator, which we denote by
\begin{equation}
\label{eq:mobeqn}
 \ddt{\V{X}}(s)-\V{U}_0\left(\V{X}(s)\right) \eqd \left(\Lop{M}\left[\V{X}(\cdot)\right]\V{f}(\cdot)\right)(s)
\end{equation}
which determines the velocity (relative to the background flow) on the centerline of each filament. This mobility equation can be closed by defining a constitutive equation for the fiber force densities $\V{f}$, which we do in the next section.

\iffalse
In sum, the overall SBT mobility on fiber $i$ is given by, 
\begin{flalign}
& \V{U}(s_i)-\V{U}_0(\V{X}(s_i)) :=(\M{M}\V{f})(s_i) := ((\M{M}_L+\M{M}_{NL})\V{f})(s_i)\\[2 pt] \nonumber &\frac{1}{8\pi\mu}\left(c(s_i)(\V{I}+\V{X}_s(s_i)\V{X}_s(s_i)) +  (\V{I}-3\V{X}_s(s_i)\V{X}_s(s_i))\right)\V{f}^i(s_i)\\[4 pt]
& \nonumber + \frac{1}{8\pi\mu}\int_{0}^L \left(\frac{\V{I}+\hat{\V{R}}(s_i,s')\hat{\V{R}}(s_i,s')}{\norm{\V{R}(s_i,s')}}\right) \V{f}^i(s') -\left(\frac{\V{I}+\V{X}_s(s_i)\V{X}_s(s_i)}{|s_i-s'|}\right) \V{f}^i(s_i) \, ds' \\[4 pt]
\nonumber
& + \frac{1}{8\pi\mu}\sum_{j \neq i} \int_0^L \left(\frac{\V{I}+\hat{\V{R}}(s_i,s_j)\hat{\V{R}}(s_i,s_j)}{\norm{\V{R}(s_i,s_j)}} + (\epsilon L)^2 \frac{\V{I}-3\hat{\V{R}}(s_i,s_j)\hat{\V{R}}(s_i,s_j)}{\norm{\V{R}(s_i,s_j)}^3}\right)\V{f}^j(s_j) \, ds_j. 
\end{flalign}
The velocity on fiber $i$ can be broken down into three parts: the leading order, purely local, term is the first line in Eq.\ \eqref{eq:fibevcont} and is denoted as $\M{M}_L \V{f}$. The second line is the singular finite part integral that gives the velocity contribution from the rest of filament $i$ to the velocity at $s_i$, and the third line is the contribution from all other fibers to the velocity at filament $i$. The second and third lines are nonlocal integrals, and we denote them together as $\M{M}_{NL} \V{f}$. 
\fi 



\section{Inextensible filaments}
In this paper, we consider inextensible filaments which can bend. Twist elasticity is neglected, as it has been in previous work on biological filaments \cite{ehssan17}. At each instant in time, fibers resist bending with bending force density (per unit length) $\V{f}^E\left(\V{X}\right)$. Inextensibility can be enforced by introducing Langrange multiplier force densities $\V{\lambda}\left(\V{X}\right)$. Thus the PDE that we need to solve is given by (dropping the explicit dependence on $s$), 
\begin{equation}
\label{eq:fibPDE}
\frac{\partial \V{X}}{\partial t} - \V{U}_0(\V{X}) = \Lop{M}\left[\V{X}\right] \left(\V{f}^E\left(\V{X}\right)+\V{\lambda}\left(\V{X}\right)\right),
\end{equation}
where the mobility operator $\Lop{M}$ is defined in Eq.\ \eqref{eq:fibevcont}. The fibers are assumed to be free, and are therefore subject to the boundary conditions \cite{ts04}
\begin{equation}
\label{eq:frBCs}
\V{X}_{ss}\left(s=0,L\right)=\V{X}_{sss}\left(s=0,L\right)=\bm{0}.
\end{equation}
Here by $\V{X}_{ss}$ we mean $\left(\ind{\V{X}}{i}\right)_{ss}$ for all fibers $i$ (and similarly for any number of $s$ derivatives). 

The fibers are also constrained to be inextensible, so that 
\begin{equation}
\label{eq:inex}
\V{X}_s(s,t) \cdot \V{X}_s(s,t)=1 
\end{equation}
for all times $t$. In summary, we solve Eq.\ \eqref{eq:fibPDE} subject to boundary conditions in Eq.\ \eqref{eq:frBCs} and constraints in Eq.\ \eqref{eq:inex}. We need to specify additional conditions to make the solution unique, as we discuss next. 

We first require a constitutive equation for the bending force density on the fibers. We use the Euler beam model, so that the bending force is given by
\begin{equation}
\label{eq:bforce}
    \V{f}^E=-E\V{X}_{ssss}\eqd \Lop{F}\V{X},
\end{equation}
where the linear operation $\Lop{F}\V{X}$ gives $ \V{f}^E$ with proper treatment of the ``free fiber'' boundary conditions in Eq.\ \eqref{eq:frBCs}. See Section \ref{sec:rsc} for details on how we construct a discrete form of $\Lop{F}$. 

It is easy to see that the boundary conditions in Eq.\ \eqref{eq:frBCs} cause the total force and torque on each fiber due to $\V{f}^E$ to be zero, 
\begin{gather}
\label{eq:totfe}
\int_{0}^{L} \V{f}^E \, ds = -E\V{X}_{sss}\Big \rvert^{L}_{0} = \V{0}, \quad \text{and}\\[2 pt]
\left(\int_{0}^{L} \V{f}^E \times \V{X} \, ds\right)^\ell = -E\int_{0}^{L} X^j X^k_{ssss} - X^k X^j_{ssss} \, ds = -E\int_{0}^{L} X^j_{ss}  X^k_{ss} -X^k_{ss} X^j _{ss} = 0. 
\end{gather}
Here superscripts denote vector components and $(j,k,\ell)$ is a cyclic permutation of $(1,2,3)$. In the torque equation, the free fiber boundary conditions lead to the cancellation of boundary terms that arise in integration by parts.

\subsection{Traditional formulation}
In the traditional formulation of inextensibility \cite{ts04}, Eq.\ \eqref{eq:inex} is differentiated with respect to time. Then, $s$ and $t$ derivatives are interchanged to yield
\begin{gather}
\label{eq:inexdt}
\left(\ddt{\Xs}\right)_s \cdot \Xs = 0. 
\end{gather}
In \cite{ts04}, the system was closed by substituting Eq.\ \eqref{eq:mobeqn} into Eq.\ \eqref{eq:inexdt}, where the total fiber force density is given by $\V{f}=\V{f}^E + \V{\lambda}$. By assuming that $\V{\lambda}=(T\Xs)_s$, where $T(s)$ is the scalar line tension, we obtain the \textit{line tension equation} of \cite{ts04}
\begin{equation}
\label{eq:lineT}
\left(\Lop{M}\left[\V{X}\right]\left(\Lop{F}\V{X}+(T\Xs)_s\right)\right)_s  \cdot \V{X}_s = 0. 
\end{equation}
While this equation is linear in $T$, it is highly nonlinear in $\V{X}$, since the operation $\Lop{F}\V{X}$ gives fourth derivatives of $\V{X}$. Even in the absence of any nonlocal interactions (i.e. if $\Lop{M}=\ML$) and zero background flow ($\V{U}_0=\V{0}$), the line tension equation still has terms of the form $\V{X}_{sss}\cdot \V{X}_{sss}$ (see \cite[~Eq. (13)]{ts04}). Because Eq.\ \eqref{eq:lineT} is a boundary value problem enforced pointwise along the fiber, we refer to it as a \textit{strong formulation of inextensibility}. 

\subsection{New formulation \label{sec:geo}}
In our approach, the variable of interest is $\Xs(s,t)$, rather than $\V{X}(s,t)$, which can be obtained by integration modulo a constant. Considering the evolution of $\Xs(s,t)$, Eq.\ \eqref{eq:inexdt} implies that 
\begin{equation}
\label{eq:omegadef}
\frac{\partial \Xs}{\partial t}(s,t) = \V{\Omega}(s, t) \times \Xs(s,t), 
\end{equation}
i.e. that the fiber evolution can be thought of as rotations of $\Xs$ on the unit sphere. Now, at each fiber point, we uniquely define an orthonormal coordinate system using Euler angles $\theta(s,t)$ and $\phi(s,t)$. We represent the unit tangent vector $\Xs(s,t)$ as
\begin{equation}
\Xs(s,t)=\Xs(\theta(s,t),\phi(s,t)) = \begin{pmatrix} \cos{\left(\theta(s,t)\right)} \cos{\left(\phi(s,t)\right)}\\[2 pt] \sin{\left(\theta(s,t)\right)} \cos{\left(\phi(s,t)\right)} \\[2 pt] \sin{\left(\phi(s,t)\right)} \end{pmatrix}.
\end{equation}
We define $\theta$ to be single-valued at $\phi=\pi/2$ by setting $\theta \left(\phi=\pm \pi/2\right)=0$. A choice of normal vectors that are always orthonormal to $\Xs$ on the unit sphere is
\begin{equation}
\label{eq:nangles}
\V{n}_1 =  \begin{pmatrix} -\sin{\theta}\\[2 pt] \cos{\theta}\\[2 pt]0 \end{pmatrix} \qquad \V{n}_2 =  \begin{pmatrix} -\cos{\theta} \sin{\phi}\\[2 pt] -\sin{\theta} \sin{\phi} \\[2 pt] \cos{\phi} \end{pmatrix}. 
\end{equation}
Other choices are also possible. Because $\V{n}_1$ and $\V{n}_2$ can be determined uniquely from $\Xs$, we refer to their dependencies from here forward as $\V{n}_j[\Xs(s,t)]$, for $j=1, 2$. Since $\theta$ is single-valued at $\phi=\pi/2$, each component of the orthonormal coordinate system $(\Xs,\V{n}_1,\V{n}_2)$ is smooth when $\Xs$ is smooth. 

Because $\Xs \times \Xs=\V{0}$, $\V{\Omega}(s,t)$ can be represented by linear combinations of $\V{n}_1$ and $\V{n_2}$. We let 
\begin{equation}
\label{eq:omdef}
\V{\Omega}(s,t) \eqd \V{\Omega}\left(\Xs(s,t),\V{g}(s,t)\right) \eqd g_1(s,t)\V{n}_2[\Xs(x,t)]-g_2(s,t)\V{n}_1[\Xs(s,t)],  
\end{equation}
where $g_1(s,t)$ and $g_2(s,t)$ are two specific unknown functions in $L^2:[0,L]$. Eq.\ \eqref{eq:omdef} implies that, by the right-handedness of the coordinate system $(\V{X}_s, \V{n}_1, \V{n}_2)$, 
\begin{equation}
\label{eq:Xsupdate}
\ddt{\Xs} = \V{\Omega}\left(\Xs(s,t),\V{g}(s,t)\right) \times \Xs(s,t) = g_1(s,t)\V{n}_1[\Xs(s,t)] + g_2(s,t)\V{n}_2[\Xs(s,t)]. 
\end{equation}
%The position of the fiber $\V{X}$ can be recovered by 
%\begin{equation}
%\label{eq:XfromXs}
%\V{X}(s)= \V{X}(0)+\int_0^s \V{X}_s(\theta(s),\phi(s)) \, ds, 
%\end{equation}
The velocity of any given fiber centerline can now be written as 
\begin{equation}
\label{eq:velfib}
\ddt{\V{X}}(s,t) = \V{U}(t)+\int_0^s  \sum_{j=1}^2 g_j(s',t)\V{n}_j[\Xs(s',t)] \,ds' . 
\end{equation}
where $\V{U}(t)$ is a rigid body translation. 

%Here we have defined a linear operator $\bm{K}$ acting on a vector $\bm{g}=(g_1,g_2,\bm{U})$. 
%Substituting Eq.\ \eqref{eq:defk} into Eq.\ \eqref{eq:fibevcont}, we have 
%\begin{equation}
%\label{eq:svcont}
%\left(\bm{K}\bm{g}\right)(s) = \bm{M}\left(\bm{f}^E+\bm{\lambda}\right)+\bm{U}_0(\bm{X}). 
%\end{equation}
%Therefore, can be viewed as a rotational velocity for $\V{X}_s$ on the unit sphere. The notation here indicates that $\V{\alpha}$ determines $g_1$ and $g_2$ and therefore $\V{\Omega}$. We will use this to update the configuration in our temporal integration schemes by rotating the tangent vector $\V{X}_s$ and then computing $\V{X}$ using \eqref{eq:XfromXs}; this strictly enforces inextensibility.
%Because our formulation involves integral rather than differential equations, we refer to it as a weak formulation of inextensibility. In fact, taking the derivative of Eq.\ \eqref{eq:svcont} with respect to $s$ and taking the inner product with $\bm{X}_s$ gives the line tension equation of \cite{ts04}. 

\subsection{Principle of virtual work}
So far, we have only laid out the new geometry we use to evolve the filament. In fact, the formulation of Section \ref{sec:geo} can still be coupled to a line tension equation to obtain the fiber velocity. We choose to close our formulation differently, in the process eliminating the need for an auxiliary line tension equation. 

The principle of virtual work states that the constraint forces $\V{\lambda}(s,t)$ do no work for any choice of $\V{g}(s,t)$ \cite{varibook}. Because this constraint holds for all $t$, we drop for the moment the explicit dependence on $t$ in the notation. To impose this constraint, we use the $L^2$ inner product to compute the total dissipated power on the fluid from $\V{\lambda}$. 
\begin{align}
\label{eq:15}
\mathcal{P}& =\Bigg{\langle} \V{\lambda},\ddt{\V{X}} \Bigg{\rangle} = \int_0^L \left(\V{U}+\int_0^{s'} g_1(s) \V{n}_1[\Xs(s)] + g_2(s)\V{n}_2[\Xs(s)] \,ds\right) \cdot \V{\lambda}\left(s'\right) \, ds'\\[2 pt] 
\label{eq:16}
& = \V{U} \cdot \int_0^L \V{\lambda}\left(s'\right) \, ds' +\int _0^L ds \int_{s}^L \left(g_1(s)\V{n}_1[\Xs(s)] + g_2(s) \V{n}_2[\Xs(s)] \right) \cdot \V{\lambda}(s') \, ds'\\[2 pt]
\label{eq:Kstarcont}
& = \V{U} \cdot \int_0^L \V{\lambda}\left(s'\right) \, ds' + \int_0^L \left(g_1(s)\V{n}_1[\Xs(s)] + g_2(s) \V{n}_2[\Xs(s)] \right) \cdot  \left(\int_{s}^L  \V{\lambda}(s') \, ds'\right) \, ds=0 
\end{align}
Here Eq.\ \eqref{eq:16} is obtained from Eq.\ \eqref{eq:15} via a simple change of integration variables. Since Eq.\ \eqref{eq:Kstarcont} must hold for all $\bm{U}$ and all sufficiently smooth $g_1$ and $g_2$, we must have that, for all $s$, 
\begin{equation}
\label{eq:noworkcont}
\begin{pmatrix} \left(\int_s^L \V{\lambda}(s')\, ds'\right) \cdot \V{n}_1[\Xs(s)]\\[2 pt] \left(\int_s^L \V{\lambda}(s')\, ds'\right) \cdot\V{n}_2[\Xs(s)]\\[2 pt] \int_0^L \V{\lambda}(s') ds' \end{pmatrix} = \begin{pmatrix} 0 \\[2 pt] 0\\[2 pt] \bm{0}\end{pmatrix}. 
\end{equation}
This constraint can be used to obtain a closed form for $\V{\lambda}$. The first and second components of Eq.\ \eqref{eq:noworkcont} taken together tell us that $\left(\int_{s}^L  \V{\lambda}\left(s'\right) \, ds'\right)$ is orthogonal to both normal vectors. Therefore, $\left(\int_{s}^L  \V{\lambda}\left(s'\right) \, ds'\right)$ is in the direction of $\V{X}_s(s)$ and can be written as
\begin{equation}
\label{eq:tsalm}
\int_{s}^L  \bm{\lambda}\left(s'\right) \, ds' = -T(s)\Xs,
\end{equation}
for some scalar function $T(s)$ with $T(s=L)=0$. Differentiating both sides of Eq.\ \eqref{eq:tsalm} with respect to $s$, we obtain 
\begin{equation}
\label{eq:lamval}
\V{\lambda}(s) = \left(T(s)\Xs\right)_s, 
\end{equation}
which is the form of Tornberg and Shelley \cite{ts04}. Thus our derivation shows that the form of $\V{\lambda}$ taken in \cite{ts04} is equivalent to the principle that the constraint forces perform no virtual work \cite{varibook}. 

Now, returning to the third constraint in Eq.\ \eqref{eq:noworkcont}, $\int_0^L \V{\lambda}(s) \, ds=\V{0}$, and substituting Eq.\ \eqref{eq:lamval} for $\V{\lambda}$, we obtain 
\begin{equation}
\label{eq:BCT}
T(L)\Xs(L) - T(0)\Xs(0)=\V{0}. 
\end{equation}
Since $T(L)=0$, Eq.\ \eqref{eq:BCT} implies that $T(0)=0$ as well, since neither of the tangent vectors is identically 0. So we obtain $T(0)=T(L)=0$, which is exactly the boundary condition for the line tension equation in \cite{ts04}. 

In this sense, the constraint equations \eqref{eq:15}$-$\eqref{eq:noworkcont} are all equivalent to the line tension equation used in prior work \cite{ts04}. Because we showed the equivalence by enforcing Eq.\ \eqref{eq:Kstarcont} ``for every choice of $g_1(s)$ and $g_2(s)$,'' we refer to our new formulation as a \textit{weak formulation of inextensibility}. In the next section, we show how choosing a suitable basis for $g_1(s)$ and $g_2(s)$  leads to a method with a Galerkin-like feel.  

\subsection{$L^2$ weak formulation}
\label{sec:numinex}
In this section, we introduce an $L^2$ weak formulation that will later lead to a novel numerical method.

The key idea is to expand the unknown functions $g_1(s,t)$ and $g_2(s,t)$ as, 
\begin{equation}
\label{eq:basis}
g_j(s,t) = \sum_k \alpha_{jk}(t) T_k(s), \quad \text{for $j=1, 2$}, 
\end{equation}
where $T_k(s)$ are sufficiently smooth scalar-valued basis functions for $L^2:[0,L]$. Here we have re-introduced the dependence on time $t$ to show that the basis functions are constant in time while their coefficients vary. 

Substituting the expansion of Eq.\ \eqref{eq:basis} into Eq.\ \eqref{eq:velfib}, we obtain
\begin{equation}
\label{eq:du}
\ddt{\V{X}}(s,t) =\bm{U}(t) + \int_0^s \sum_{j=1}^2\sum_k \alpha_{jk}(t) T_k\left(s'\right) \V{n}_j[\Xs(s',t)] \, ds' \eqd (\Lop{K}\left[\V{X}(\cdot,t)\right]\V{\alpha}(t))(s),
\end{equation}
where the linear operator $\Lop{K}\left[\V{X}(\cdot,t)\right]$ acts on $\V{\alpha}(t)=(\alpha_{jk}(t),\bm{U}(t))$ to give an inextensible velocity on the filament centerline ($\V{\alpha}$ parameterizes inextensible fiber motions). Note the dependence of $\Lop{K}$ on $\V{X}$ since $\Lop{K}$ involves the normal vectors $\V{n}_1$ and $\V{n}_2$. Using Eq.\ \eqref{eq:du}, The dynamical equation, \eqref{eq:mobeqn}, now becomes
\begin{equation}
\label{eq:veleqn}
(\Lop{K}\left[\V{X}(\cdot,t)\right]\V{\alpha}(t))(s) = \V{U}_0\left(\V{X}(s,t),t\right) + \left(\Lop{M}\left[\V{X}(\cdot,t)\right]\left(\V{f}^E(\cdot,t) +\V{\lambda}(\cdot,t)\right)\right)(s)
\end{equation}

This new dynamical equation, \eqref{eq:veleqn} is supplemented by enforcing Eq.\ \eqref{eq:noworkcont} in an $L^2$ weak sense. We begin by substituting the representation of $g_j$ in Eq.\ \eqref{eq:basis} into the power equation, \eqref{eq:15}, to obtain (again dropping the time dependence in the notation)
\begin{align}
\label{eq:26}
\mathcal{P} & = \bigg{\langle} \V{\lambda}(\cdot),\Lop{K}[\V{X}(\cdot)]\V{\alpha}\bigg{\rangle}= 0 \\[2 pt]
\label{eq:Kstardef}
\mathcal{P} & = \V{U} \cdot \int_0^L \V{\lambda}(s) \, ds + \int_0^L \left(\int_0^{s} \sum_{j=1}^2\sum_{k} \alpha_{jk} T_k(s') \V{n}_j[\Xs(s')] \, ds'\right)\cdot \V{\lambda}\left(s\right) \, ds\\[2 pt]
\nonumber & \eqd \bigg{\langle} \Lop{K}^*[\V{X}(\cdot)]\V{\lambda}(\cdot), \V{\alpha}\bigg{\rangle} =0. 
\end{align}
Eq.\ \eqref{eq:Kstardef} formally defines the $L^2$ adjoint of $\Lop{K}$ as $\Lop{K}^*$. Since the power from the constraint forces must be zero for any inextensible motion (any $\V{\alpha}$), we define $\Lop{K}^*$ by setting each term of Eq. \eqref{eq:Kstardef} to zero. That is,
\begin{equation}
\label{eq:noworkcontL2}
\Lop{K}^*[\V{X}(\cdot)]\V{\lambda}(\cdot):=\begin{pmatrix} \int_0^L \left(\int_0^{s} T_k(s') \V{n}_1[\Xs(s')] \, ds'\right)\cdot \bm{\lambda}(s) \, ds\\[2 pt] \int_0^L \left(\int_0^{s} T_k(s') \V{n}_2[\Xs(s')] \, ds'\right)\cdot \bm{\lambda}(s) \, ds\\[2 pt] \int_0^L \bm{\lambda}(s) \, ds \end{pmatrix} = \begin{pmatrix} 0 \\[2 pt] 0\\[2 pt] \V{0}\end{pmatrix}, 
\end{equation}
where the first two vector components hold for all $k$ and the last vector component holds in each of the three directions.

A saddle-point system can now be written for $\bm{\lambda}$ and $\bm{\alpha}$ by combining Eqs.\ \eqref{eq:veleqn} and \eqref{eq:noworkcontL2}. Dropping all dependencies on $s$ and $t$ and setting $\V{f}^E=\Lop{F}\V{X}$, we have
\begin{equation}
\label{eq:saddleL2}
    \begin{pmatrix}
    -\Lop{M}[\V{X}] & \Lop{K}[\V{X}]\\[4 pt]
    \Lop{K}^*[\V{X}] & \bm{0}
    \end{pmatrix}
    \begin{pmatrix} 
    \V{\lambda}\\[4 pt]
    \V{\alpha}
    \end{pmatrix} =  \begin{pmatrix} 
    \Lop{M}[\V{X}]\left(\Lop{F}\V{X}\right)+\V{U}_0\left(\V{X}\right)\\[4 pt]
    \bm{0}\end{pmatrix}.
\end{equation}

In continuum, solving this system for $\V{\lambda}$ and $\V{\alpha}$ is equivalent to solving the line tension equation for $T$ and letting $\V{\lambda}=(T\V{X}_s)_s$. Numerically, Eq.\ \eqref{eq:saddleL2} allows us to avoid solving the problematic line tension equation. \cmt{Summary of eqns?}

\iffalse
Now, we can eliminate $\bm{\lambda}$ by a Schur complement approach. Applying the formal (psuedo)inverse of $\bm{M}$ to both sides of Eq.\ \eqref{eq:veleq} and then applying $\bm{K}^*$ to both sides, we obtain
\begin{equation}
\left(\left(\bm{K}^*\bm{M}^{-1} \bm{K}\right)\bm{\alpha}\right)(s) = \bm{K}^*\left(\bm{f}^E+\bm{M}^{-1}\bm{U}_0\right)=\bm{K}^*\left(\bm{L}\bm{X}+\bm{M}^{-1}\bm{U}_0\right),
\end{equation}
where we have used Eq.\ \eqref{eq:noworkcontL2} to eliminate $\bm{\lambda}$. If we now apply the (pseudo)inverse of the operator $\bm{K}^*\bm{M}^{-1}\bm{K}$, then apply $\bm{K}$ on the left to both sides, we obtain a solution for the velocity of the fiber centerline of the form
\begin{equation}
\frac{\partial \bm{X}}{\partial t}(s) = \left(\bm{K}\bm{\alpha}\right)(s) = \left(\bm{K}\left(\bm{K}^*\bm{M}^{-1}\bm{K}\right)^{-1}\bm{K}^*\right)\left(\bm{L}\bm{X}+\bm{M}^{-1}\bm{U}_0\right):
=\bm{N}\left(\bm{L}\bm{X}+\bm{M}^{-1}\bm{U}_0\right). 
\end{equation}
\fi

\section{Numerical Methods}
This section is devoted to our spatial and temporal discretizations of Eq.\ \eqref{eq:saddleL2} (solving for $\V{\alpha}$) and Eq.\ \eqref{eq:defk} (updating the fiber tangent vector and position once $\V{\alpha}$ is known). We begin by laying out our spectral choice of basis functions $\phi_k$ and spectral spatial discretization in Section \ref{sec:discspat}. In Section \ref{sec:tint}, we discuss how our second-order temporal discretization avoids having to perform nonlinear solves and requires only a single evaluation of the nonlocal hydrodynamics for each timestep. We conclude this section by discussing how we actually evaluate each of the nonlocal terms to 3 digits of accuracy. 

\subsection{Spectral discretization \label{sec:discspat}}
\subsubsection{Basis for $L^2$}
We will use a Chebyshev basis to discretize the fiber centerline. That is, 
\begin{equation}
\label{eq:basisD}
g_j(s) = \sum_{k=0}^{N-2} \alpha_{jk} T_k(s), 
\end{equation}
where $T_k(s)$ is the Chebyshev polynomial of the first kind of degree $k$ on $[0,L]$. 

The choice of $N-2$ for the maximum summation index is a necessary (but not sufficient) condition that makes the representation $\V{U}(s) = \M{K}\V{\alpha}$ unique on an $N$ point Chebyshev grid. To see this, suppose that Chebyshev polynomials of degree $N-1$ or higher were used in Eq.\ \eqref{eq:basisD}. Then the integration in Eq.\ \eqref{eq:du} results in Chebyshev polynomials of degree $N$ or higher contributing to the velocity. A degree $N$ polynomial can be zero at all $N$ nodes without being identically zero, and the representation is not unique. Thus the summation being from $k=0$ to $N-2$ in Eq.\ \eqref{eq:basisD} is a necessary condition for $\M{K}\V{\alpha}$ to give a unique representation of the fiber velocity. This is not a sufficient condition, however, since in practice $\V{n}_j$ is also a polynomial function of $s$ and $\int_0^s T_k(s') \V{n}_j(s') \, ds'$ could still be zero on a grid of size $N$. This means that $\M{K}\V{\alpha}$ could be zero with at least one nonzero $\alpha$ value, which means that $\V{\alpha}$ does not uniquely give $\V{U}$.

\subsubsection{Spatial discretization}

Because we use a collocation discretization, the fiber is discretized at nodes $s_i$, $i=1, \dots N$, where in our case $s_i$ is a node on a type 1 Chebyshev grid (i.e. a grid that does not include the endpoints; the reason for this choice is discussed in Section \ref{sec:rsc}). We then define an operator $\intmat$ that evaluates the integrals in Eq.\ \eqref{eq:du} with $\phi_k=T_k$ via some quadrature scheme. In the Chebyshev discretization, this must be done with proper anti-aliasing. In our discretization, the operator $\left(\intmat\left(T_k(\cdot) \V{n}_j(\cdot)\right)\right)(s)$ upsamples the functions $T_k$ and $\V{X}_s$ to a $2N$ grid. On the $2N$ grid, $\intmat$ then computes the normal vectors via Eq.\ \eqref{eq:nangles}, performs multiplication with $T_k$, and applies the pseudo-inverse of the Chebyshev differentiation matrix, $\M{D}_{2N}^\dagger$. $\intmat$ then downsamples this result from the $2N$ grid to the original $N$ grid. 

\subsubsection{Fully discrete linear system}
The spatially discrete form of Eq.\ \eqref{eq:du} is now given by
\begin{gather}
\label{eq:dvel}
\frac{\partial \bm{X}}{\partial t}\left(s_i\right)=  \left(\bm{M}(\bm{\lambda} +\bm{L}\bm{X})\right)(s_i) +\bm{U}_0(\bm{X}(s_i))= \left(\bm{K}\bm{\alpha}\right)(s_i) = \bm{U} +\sum_{j=1}^2\sum_{k=0}^{N-2} \alpha_{jk} \left(\intmat \left(T_k(\cdot ) \bm{n}_j(\cdot)\right)\right)(s_i)
\end{gather}

It is now straightforward to discretize the no work constraint in Eq.\ \eqref{eq:noworkcontL2}. We define a matrix $\bm{I}^*$ that integrates a (vector or scalar) function using Clenshaw-Curtis quadrature on the type 1 Chebyshev grid. Then the fully discrete form of Eq.\ \eqref{eq:noworkcontL2} is 
\begin{equation}
\label{eq:noworkcontFD}
\bm{K}^* \bm{\lambda}=\begin{pmatrix} \bm{I}^* \left(\left(\intmat(T_k(\cdot) \bm{n}_1(\cdot))\right) \cdot \bm{\lambda}\right) \\[2 pt] \bm{I}^* \left(\left(\intmat(T_k(\cdot) \bm{n}_2(\cdot))\right) \cdot \bm{\lambda}\right)\\[2 pt] \bm{I}^*\bm{\lambda} \end{pmatrix} = \begin{pmatrix} 0 \\[2 pt] 0\\[2 pt] \bm{0}\end{pmatrix},
\end{equation}
which must hold for all $k=0, 1, \dots, N-2$. 

Thus the final saddle-point system of equations for $\bm{\lambda}$ and $\bm{\alpha}$ (the discrete form of Eq.\ \eqref{eq:saddleL2}) can be written by combining Eqs.\ \eqref{eq:dvel} and \eqref{eq:noworkcontFD}, 
\begin{equation}
\label{eq:saddlept}
    \begin{pmatrix}
    -\bm{M} & \bm{K}\\[4 pt]
    \bm{K}^* & \bm{0}
    \end{pmatrix}
    \begin{pmatrix} 
    \bm{\lambda}\\[4 pt]
    \bm{\alpha}\\[4 pt]
    \end{pmatrix} =  \begin{pmatrix} 
    \bm{M}\bm{L}\bm{X}+\bm{U}_0\\[4 pt]
    \begin{pmatrix} \bm{0}\\[4 pt]
    -\bm{I}^* \bm{L}\bm{X} \end{pmatrix}
    \end{pmatrix}.
\end{equation}
This system, which has an obvious saddle-point structure, is not invertible generally because the representation $\bm{K}\bm{\alpha}$ is not necessarily unique. We therefore solve the system in the least squares sense with a tolerance of $10^{-6}$.

Note that in Eq.\ \eqref{eq:saddlept}, the matrices $\bm{M}$ and $\bm{K}$ are functions of $\bm{X}$. That is, $\bm{M}=\bm{M}(\bm{X})$ and $\bm{K}=\bm{K}(\bm{X})$. Finally, observe that in Eq.\ \eqref{eq:saddlept}, we enforce the third component of Eq.\ \eqref{eq:noworkcontFD} up to discretization errors in $\int \bm{f}^E(s) \, ds \approx \bm{I}^*\bm{f}^E=\bm{I}^*\M{L}\V{X}$. Although $\int \bm{f}^E(s) \, ds=\bm{0}$ in the continuous case, this does not necessarily hold discretely. We therefore keep the term $\bm{I}^*\M{L}\V{X}$ in our discretization to enforce the condition that the \textit{total force on the fiber is zero exactly in the discrete setting}.

%Our formulation of the problem is based on using the computed values of $\V{\alpha}$ to evolve $\V{X}_s$ in Eq.\ \eqref{eq:Xsupdate}, and then obtaining $\V{X}$ by integration (this entire process is Eq.\ \eqref{eq:defk}). This theory is of course predicated on the smoothness of $\V{n}_1(s)$ and $\V{n}_2(s)$. In Eq.\ \eqref{eq:nangles},. \cmt{(This paragraph is probably out of place.)}

\iffalse
\textit{Uniqueness of representation}. Next we show why the upper bound on the sum in Eq.\ \eqref{eq:basisD} is $N-2$ via studying the null space of the matrix $\bm{K}$. If $\bm{K}\bm{\alpha}(s_i)=\bm{0}$ for all $i = 1, \dots N$, we look for a condition that ensures $\bm{\alpha}=\bm{0}$. Without loss of generality, suppose that we choose $\bm{n}_1$ to have a zero entry at position $p$ and $\bm{n}_2$ to have a non-zero entry at position $p$. Then we can write the $p$th entry of $\bm{K}\bm{\alpha}$ at $s_i$ as 
\begin{align}
\left(\left(\bm{K}\bm{\alpha}\right)(s_i)\right)^p & = \bm{U} + \sum_{k=0}^{N-2} \bm{D}^+ \left(\alpha_{1k} T_k \cdot 0\right)(s_i) +  \bm{D}^+ \left(\alpha_{2k} T_k n_2^p\right)(s_i)\\[4 pt] & =\bm{U} + \sum_{k=0}^{N-2} \bm{D}^+ \left(\alpha_{2k} T_k n_2^p\right)(s_i) =0
\end{align}
where $n_2^p(s)$ is some nonzero function of $s$. So if $\left(\left(\bm{K}\bm{\alpha}\right)(s_i)\right)^p=0$ for all $i$, then $\left(\left(\bm{K}\bm{\alpha}\right)(s)\right)^p$ has $N$ zeros.  So a necessary condition for an empty null space of $\bm{K}$ is that $\left(\left(\bm{K}\bm{\alpha}\right)(s)\right)^p$ have $N-1$ zeros or less. Because of the integration operator $\bm{D}^+$, this means we can only include polynomials modes up to $N-2$. Note that this is a \textit{necessary} condition, not a suficient one, since in practice we cannot know the form of the normal vectors (those could be high order polynomials). 

As in the continuous case, since this equation must hold for every feasible motion $\bm{U}$ and $\bm{\alpha}$, each term in Eq.\ \eqref{eq:power} must be zero. Let us focus on the last term for now. Discretizing the integral using some quadrature rule at $s_i$ with weights $w_i$,
\begin{align}
    \mathcal{P}_2 &= \sum_{i=1}^{N} \sum_{j=1}^2\sum_{k=0}^{N-2} \alpha_{jk}\bm{D}^+\left(\phi_k \bm{n}_j\right)(s_i) \cdot \bm{\lambda}(s_i) w_i \\[4 pt]
& = \sum_{k=0}^{N-2} \sum_{j=1}^2 \alpha_{jk} \sum_{i=1}^{N} \bm{D}^+\left(\phi_k \bm{n_j}\right)(s_i) \cdot \bm{\lambda}(s_i) w_i=0, 
\end{align}
which must hold for any $\alpha_{jk}$. In particular, it must hold for $\alpha_{jk}=\delta_{1j} \alpha_k$ and $\alpha_{jk}=\delta_{2j} \alpha_k$, and so we have that \textit{for each} $j$ \textit{and} $k$, 
\begin{equation}
\label{eq:kstardisc}
    \sum_{i=1}^{N}\bm{D}^+\left(\bm{n}_j \phi_k\right)(s_i) \, \cdot \bm{\lambda}(s_i) w_i:=\bm{K}^*\bm{\lambda}=0.
\end{equation}
Eq.\ \eqref{eq:kstardisc} defines the bulk of the adjoint condition on $\bm{\lambda}$. It still remains to enforce the first part of Eq.\ \eqref{eq:power}. 

Since $\bm{U}$ is an arbitrary constant, the first term in Eq.\ \eqref{eq:power} can be discretized as
\begin{equation}
\label{eq:IT}
   \bm{0} = \int \bm{\lambda}(s) \, ds \approx \sum_{i=1}^{N} \bm{\lambda}(s_i) w_i := \bm{I}^*\bm{\lambda},
\end{equation}
where we have defined the discrete integration matrix $\bm{I}^*$ which takes a definite integral of a scalar function (whose values are given as a vector) on $[0,L]$. 
\fi

\subsubsection{Determining the elastic forces}
\label{sec:rsc}
The final order of business in the spatial discretization is to compute $\bm{f}^E=\M{L}\V X$ accurately and with the correct boundary conditions. We use rectangular spectral collocation \cite{tref17, dhale15} to determine the operator $\bm{L}$. We follow the convention of \cite{dhale15} and solve Eq.\ \eqref{eq:saddlept} on a type 1 Chebyshev grid with $N$ points (i.e. the grid where the PDE is posed \textit{does not} include the boundary points). The boundary conditions are imposed by upsampling the relevant quantities to a type 2 Chebyshev grid (that includes the endpoints) with $\tilde{N}=N+4$ points, since there are 4 BCs. %This procedure is analogous to using ghost cells in finite difference methods. %For the rest of this report, any quantity defined on the type 2 grid is marked with a tilde. 
Given the values of $\bm{X}$ on a type 1 Chebyshev grid with $N$ points, there is a unique configuration $\tilde{\bm{X}}$ on the type 2 grid that satisfies
\begin{equation}
\label{eq:deftilde}
\begin{pmatrix} \bm{R} \\[2 pt] \bm{B} \end{pmatrix} \tilde{\bm{X}} = \begin{pmatrix} \bm{I}_N \\[2 pt] \bm{0} \end{pmatrix} \bm{X}. 
\end{equation}
Here $\bm{R}$ is the downsampling operator that intepolates the data on the type 1 $N$ point grid from the data on an $\tilde{N}=N+4$ type 2 grid, and $\bm{B}$ is the operator that encodes the boundary conditions $\displaystyle \tilde{\bm{X}}_{ss}\left(s=0,L\right)=\tilde{\bm{X}}_{sss}\left(s=0,L\right)$ on the \textit{type 2 grid} (note that the right hand side of Eq.\ \eqref{eq:deftilde} could easily be modified to encode other boundary conditions). Because $\M{R}$ is an $N \times (N+4)$ matrix and $\M{B}$ is a $4 \times (N+4)$ matrix, the left hand side of Eq.\ \eqref{eq:deftilde} is invertible and we can therefore write 
\begin{equation}
\label{eq:getX}
\tilde{\bm{X}} = \begin{pmatrix} \bm{R} \\[2 pt] \bm{B} \end{pmatrix}^{-1} \begin{pmatrix} \bm{I}_N \\[2 pt] \bm{0} \end{pmatrix} \bm{X}= \bm{E}\bm{X}.
\end{equation}

Thus for every configuration $\bm{X}$, there is a unique function $\tilde{\bm{X}}$ on the type 2 grid that satisfies Eq.\ \eqref{eq:getX}.  In finite difference schemes, there are unique values of the ``ghost cells'' that allow the boundary stencils to satisfy the BCs to some order. Thus the rectangular spectral collocation method can be thought of as an extension of ghost cell techniques for finite difference methods.  

The function $\tilde{\bm{X}}$ can be used to compute $\bm{f}^E$ in a way consistent with the boundary conditions. That is, $\tilde{\bm{f}}^E=-E\tilde{\bm{X}}_{ssss}$ is computed on the \textit{type 2} grid and then downsampled via the operator $\bm{R}$ (analogous to using ghost cells at the boundaries to compute the fourth derivative of $\bm{X}$). We write the downsampled bending force as
\begin{equation}
\label{eq:fE}
\bm{f}^E=\bm{R}\tilde{\bm{f}}^E=-\bm{R}E\tilde{\bm{D}}^4 \tilde{\bm{X}} = -E\bm{R}\tilde{\bm{D}}^4 \bm{E}\bm{X}:=\bm{L}\bm{X}.
\end{equation}
So that we have defined the bending force on the type 1 grid, $\bm{f}^E=\bm{L}\bm{X}$. Notice that $\bm{L}$ is a \textit{constant} matrix (not a function of $\bm{X}$). It can therefore be precomputed and applied to compute $\V{f}^E$ for a given configuration $\V{X}$. 

\subsection{Temporal discretization \label{sec:tint}}
For temporal integration, we use a combination of Crank-Nicolson for the linear terms and a linear multistep method to obtain arguments for the nonlinear operations. 

\subsubsection{Solving for $\V{\alpha}$ and $\V{\lambda}$}
Recall that in Eq.\ \eqref{eq:fibevcont}, we separated the mobility into a local and nonlocal contribution. Since the local term is leading order in $\epsilon$, we treat it implicitly using Crank-Nicolson. We do the nonlocal terms explicitly in a second-order way using Adams-Bashforth style extrapolation. 

Using this discretization scheme, we need to solve the following equation for $\V{\lambda}$ at timestep $n$ 
\begin{gather}
\label{eq:CNnL}
\M{M}_L^{n+1/2,*} \left(\V{\lambda}^{n+1/2}+\frac{1}{2}\left(\left(\V{f}^E\right)^{n+1}+\left(\V{f}^E\right)^{n}\right)\right)+\M{M}_{NL}^{n+1/2,*} \left( \V{\lambda}^{n+1/2}+\left(\V{f}^E\right)^{n+1/2,*}\right) = \M{K}^{n+1/2,*}\V{\alpha},\\[2 pt]
\nonumber
\left(\M{K}^{n+1/2,*}\right)^*\V{\lambda}^{n+1/2} = \V{0}. 
\end{gather}
Here $\bm{X}^{n+1/2,*}=\frac{3}{2}\bm{X}^n-\frac{1}{2}\bm{X}^{n-1}$, and we use $\bm{X}^{n+1/2,*}$ to calculate all of the starred quantities; for example $\M{M}_{L}^{n+1/2,*}=\M{M}_L\left(\V{X}^{n+1/2,*}\right)$. 

In order to simplify the evaluation of $\left(\V{f}^E\right)^{n+1}$, we make the second order approximation 
\begin{equation}
\left(\V{f}^E\right)^{n+1}=\M{L}\left(\V{X}^n+\Delta t \M{K}\V{\alpha}\right). 
\end{equation}
This allows us to treat the local $\V{f}^E$ via Crank-Nicolson without considering the fact that we actually update the fiber by rotating $\V{X}_s$ and integrating. 

Now, it is possible to solve Eq.\ \eqref{eq:CNnL} \textit{exactly} for $\V{\lambda}^{n}$ via a fixed point iteration. That is, at step $m$ we can solve
\begin{gather}
\label{eq:CNnLFP}
\M{M}_L^{n+1/2,*} \left(\V{\lambda}^{n+1/2, m}+\frac{1}{2}\left(\left(\V{f}^E\right)^{n+1, m}+\left(\V{f}^E\right)^{n}\right)\right)+\M{M}_{NL}^{n+1/2,*} \left( \V{\lambda}^{n+1/2, m-1}+\left(\V{f}^E\right)^{n+1/2,*}\right) = \M{K}^{n+1/2,*}\V{\alpha},\\[2 pt]
\nonumber
\left(\M{K}^{n+1/2,*}\right)^*\V{\lambda}^{n+1/2, m} = \V{0}. 
\end{gather}
for $\V{\lambda}^{n+1/2, m}$. This scheme can be shown both theoretically and empirically to be second order if $\V{\lambda}$ is solved for exactly. It is, however, expensive, as it requires multiple evaluations of the nonlocal hydrodynamics per timestep. 

In our approximate scheme, we instead evaluate the nonlocal hydrodynamics using $\V{\lambda}^{n+1/2,*}=2\V{\lambda}^{n-1/2}-\V{\lambda}^{n-3/2}$, which is a second order approximation to $\V{\lambda}^{n+1/2}$. In our one step scheme, we therefore solve
\begin{gather}
\label{eq:CNnLUs}
\M{M}_L^{n+1/2,*} \left(\V{\lambda}^{n+1/2}+\frac{1}{2}\left(\left(\V{f}^E\right)^{n+1}+\left(\V{f}^E\right)^{n}\right)\right)+\M{M}_{NL}^{n+1/2,*} \left( \V{\lambda}^{n+1/2, *}+\left(\V{f}^E\right)^{n+1/2,*}\right) = \M{K}^{n+1/2,*}\V{\alpha},\\[2 pt]
\nonumber
\left(\M{K}^{n+1/2,*}\right)^*\V{\lambda}^{n+1/2} = \V{0}. 
\end{gather}
for $\V{\lambda}^{n+1/2}$. Eq.\ \eqref{eq:CNnLUs} is now a linear equation for $\V{\lambda}^{n+1/2}$ and can be viewed as one step of the fixed point iteration in Eq.\ \eqref{eq:CNnLFP}, with the initial guess $\V{\lambda}^{n+1/2,0}=\V{\lambda}^{n+1/2, *}$. Because we use an initial guess that is a combination of the previous two timesteps, we solve the problem exactly at $t=0$ and $t=\Delta t$ so that $\bm{\lambda}^{1/2}$ and $\bm{\lambda}^{3/2}$ are known exactly. After $t=\Delta t$, the nonlocal hydrodynamics need to be evaluated only once per timestep to obtain $\V{\lambda}$ and $\V{\alpha}$. 

\subsubsection{Updating $\V{X}_s$ and $\V{X}$}
Once we have computed $\V{\alpha}$, we use a discrete form of Eq.\ \eqref{eq:omegadef} to update the tangent vectors. Our goal is to rotate $\V{X}_s^n$ on the unit sphere by the vector $\V{\Omega}^{n+1/2} =  \V{\Omega}\left(\V{X}^{n+1/2,*},\V{\alpha}^{n+1/2}\right)$. Using the definition of $\V{\Omega}$ in Eq.\ \eqref{eq:omdef} and the spectral choice of basis functions, we have 
\begin{align}
\V{\Omega}^{n+1/2}(s) & = g_1(s)\V{n}_2^{n+1/2,*}(s) -g_2(s) \V{n}_1^{n+1/2, *}(s)\\[2 pt]
& = \sum_{k=0}^{N-2} \alpha_{1k}^{n+1/2} T_k(s) \V{n}_2^{n+1/2,*}(s) -\alpha_{2k}^{n+1/2} T_k(s) \V{n}_1^{n+1/2, *}(s). 
\end{align} 
Note that $\V{\Omega}^{n+1/2}$ must be computed with proper anti-aliasing. As in the computation of $\M{K}$, we upsample $\V{X}^{s,n+1/2,*}$ to a $2N$ grid and multiply by the Chebyshev polynomials $T_k$ on the upsampled grid. We then downsample the result to obtain $\V{\Omega}^{n+1/2}$ on the $N$ point grid.

Once $\V{\Omega}^{n+1/2}$ is known, we use the Rodrigues rotation formula \cite{rodrigues1840lois} to compute the rotated $\V{X}_s$. Letting $\Omega = \norm{\V{\Omega}}$ and $\hat{\V{\Omega}}=\V{\Omega}/\Omega$, we compute the rotated $\V{X}_s$ as
\begin{equation}
\V{X}_s^{n+1} = \V{X}_s^n \cos{\left(\Omega^{n+1/2}\D t\right)} + \left(\hat{\V{\Omega}}^{n+1/2} \times \V{X}_s^n \right)  \sin{\left(\Omega^{n+1/2}\D t\right)}+\hat{\V{\Omega}}^{n+1/2}\left(\hat{\V{\Omega}}^{n+1/2} \cdot \V{X}_s^n\right)\left(1-\cos{\left(\Omega^{n+1/2}\D t\right)}\right).
\end{equation}
\cmt{The last term is 0 in continuum but not discretely. Should we include it?} We then compute $\V{X}^{n+1}$ via Chebyshev integration. Specifically, we compute the Chebyshev series coefficients of $\V{X}_s^{n+1}$, apply the spectral integration matrix \cite{greengard1991spectral} to compute the Chebyshev series of $\V{X}^{n+1}$, then evaluate this series at the nodes $s_i$ (this result is off by an unknown constant). We then add a constant velocity so that the position at $s=s_1$ is the same as it would have been had the fiber been updated via adding $\Delta t \M{K}\V{\alpha}$. Specifically, we set
\begin{equation}
\V{X}^{n+1}(s_1) = \V{X}^n(s_1) + \Delta t \left(\M{K}\V{\alpha}\right)(s_1). 
\end{equation}
The positions of all other points $\V{X}(s_i)$ can then be determined from the Chebyshev series coefficients of $\V{X}(s)$ and the velocity at $s=s_1$. 

\subsection{Efficient evaluation of nonlocal SBT}
The evaluation of the $\M{M}_{NL}$ term in Eq.\ \eqref{eq:CNnLUs} can be broken down into three parts. First, we must evaluate the singular finite part integral in Eq.\ \eqref{eq:fibevcont}. Second, we must be able to evaluate the Stokeslet and Doublet kernels on other fibers with controlled accuracy, independent of how close together the fibers are. And finally, in order to efficiently simulate systems with $\mathcal{O}(100-1000)$ fibers, we must be able to efficiently sum the kernels over many fibers. These sums are quadratic in the number of fibers when done naively. 

\subsubsection{Finite part integral \label{sec:tornFP}}
Tornberg has formulated a spectrally accurate method for the nonlocal integral \cite{tornquad}. The key to this scheme is to isolate the singularity at $s'=s_i$ and remove it from the integral. We can write the nonlocal integral as
\begin{equation}
\label{eq:Jrewrite}
\M{J}[\V{f}](s_i) = \int_0^L \V{g}(s_i,s') \frac{s'-s_i}{|s'-s_i|} \, ds', 
\end{equation}
where 
\begin{equation}
\V{g}(s_i,s') = \left[ \left(\V{I}+\hat{\V{R}}(s_i,s')\hat{\V{R}}(s_i,s')\right) \frac{|s'-s_i|}{\norm{\V{R}(s_i,s')}} \V{f}(s') - \left(\V{I}+\V{X}_s(s_i) \V{X}_s(s_i)\right) \V{f}(s_i)\right] \frac{1}{s'-s_i}. 
\end{equation}
By adding and subtracting $\left(\V{I}+\V{X}_s(s_i)\V{X}_s(s_i)\right)\V{f}(s')$ inside the square bracket, $\V{g}$ can be rewritten as
\begin{equation}
\V{g}(s_i,s')=\V{g}_1(s_i,s') + \left(\V{I}+\V{X}_s(s_i)\V{X}_s(s_i)\right)\frac{\V{f}(s')-\V{f}(s_i)}{s'-s_i}, 
\end{equation}
where
\begin{equation}
\V{g}_1(s_i,s') = \left[ \left(\V{I}+\hat{\V{R}}(s_i,s')\hat{\V{R}}(s_i,s')\right) \frac{|s'-s_i|}{\norm{\V{R}(s_i,s')}} \V{f}(s') - \left(\V{I}+\V{X}_s(s_i) \V{X}_s(s_i)\right)\V{f}(s')\right] \frac{1}{s'-s_i}. 
\end{equation}
When $\V{g}$ is rewritten in this way, it has a limit as $s' \rightarrow s_i$ which can be computed via Taylor expansion around $s'=s_i$,  
\begin{equation}
\lim_{s' \to s_i} \V{g}(s_i,s') = \frac{1}{2}\left(\V{X}_s(s_i)\V{X}_{ss}(s_i)+\V{X}_{ss}\V{X}_s(s_i)\right)\V{f}(s_i) + \left(\V{I}+\V{X}_s(s_i)\V{X}_s(s_i)\right)\V{f}'(s_i). 
\end{equation}
We use this representation to compute $\V{J}[\V{f}](s_i)$ in Eq.\ \eqref{eq:Jrewrite} in the following way. Consider the $i$th component of $\V{g}$, and let $\phi=\V{g}_i$. Then we need to evaluate integrals of the form
\begin{equation}
\label{eq:specint}
I = \int_0^L \phi(s_i,s') \frac{s'-s_i}{|s'-s_i|} \, ds' = \frac{L}{2}\int_{-1}^1 \phi(\eta_i, \eta') \frac{\eta'-\eta_i}{|\eta'-\eta_i|} \, d\eta', 
\end{equation}
where the change of variables is $\eta=-1+\frac{2}{L}s$. The key to the method is now to expand $\phi$ in a monomial basis as 
\begin{equation}
\phi(\eta,\eta') = \sum_{k=0}^{N-1} c_k (\eta')^k, 
\end{equation}
where $N$ is the number of Chebyshev points. The integral in Eq.\ \eqref{eq:specint} can now be done as
\begin{equation}
\label{eq:expandmono}
I(\eta_i)= \frac{L}{2} \sum_{k=0}^{N-1} c_k \int_{-1}^1 (\eta')^k \frac{\eta'-\eta_i}{|\eta'-\eta_i|} \, d\eta' = \sum_{k=0}^{N-1} c_k q_k(\eta_i), 
\end{equation}
where 
\begin{equation}
q_k(\eta_i) = \frac{L}{2}\int_{-1}^1 (\eta')^k \frac{\eta'-\eta_i}{|\eta'-\eta_i|} = \left(\frac{L}{2}\right)\frac{1+(-1)^{k+1}-2\eta_i^{k+1}}{k+1}
\end{equation}
is known analytically. As in \cite{tornquad}, we introduce the Vandermonde matrix $\V{V}$. Let $\V{p}$ be the values of $\phi$ at the Chebyshev nodes. Then we can obtain the coefficients $\V{c}$ by solving the linear system $\V{V}\V{c}=\V{p}$. Substituting this into Eq.\ \eqref{eq:expandmono}, we have 
\begin{equation}
\label{eq:specscheme}
I(\eta_i)=\V{c}^T \V{q} = (\V{V}^{-1}\V{p})^T \V{q}(\eta_i) = \V{p}^T \left(\V{V}^{-T}\V{q}(\eta_i)\right) = \V{p}^T \V{b}(\eta_i). 
\end{equation}
Thus for each $\eta_i$, we can precompute $\left(\V{V}^{-T}\V{q}(\eta_i)\right)$ and take the inner product of this vector with the values of $\phi$. Importantly, the Vandermonde matrix must be sufficiently well-conditioned to do this calculation accurately. This means that the fiber discretization can have at most 40 points. If higher accuracy is needed, then the fiber must be split into multiple panels. 


\subsection{Near-singular integrals \label{sec:specquad}}
The contribution of fiber $j$ to the velocity at point $s_i$ on fiber $i$ is given by the line integral
\begin{gather}
\label{eq:jf1}
\V{U}_{JF}(s_i) =\int_0^L \left(\frac{\M{I}+\hat{\V{R}}(s_i,s_j)\hat{\V{R}}(s_i,s_j)}{\norm{\V{R}(s_i,s_j)}} + (\epsilon L)^2 \frac{\M{I}-3\hat{\V{R}}(s_i,s_j)\hat{\V{R}}(s_i,s_j)}{\norm{\V{R}(s_i,s_j)}^3}\right)\V{f}^j(s_j) \, ds_j. 
\end{gather}
When $\V{X}^i(s_i)$ (the \textit{target point} on fiber $i$) approaches the centerline $\V{X}^j$ of fiber $j$, this integral becomes \textit{nearly singular}, and again special quadrature schemes are needed to evaluate it accurately. We begin by rewriting Eq.\ \eqref{eq:jf1} so that the near-singularity is entirely in the denominator (i.e. by removing the hats)
\begin{gather}
\label{eq:jf2}
\V{U}_{JF}(s_i) =\int_0^L \left(\frac{\M{I}}{\norm{\V{R}(s_i,s_j)}}+\frac{\V{R}(s_i,s_j)\V{R}(s_i,s_j)+(\epsilon L)^2 \M{I}}{\norm{\V{R}(s_i,s_j)}^3} -3(\epsilon L)^2 \frac{\V{R}(s_i,s_j)\V{R}(s_i,s_j)}{\norm{\V{R}(s_i,s_j)}^5}\right)\V{f}^j(s_j) \, ds_j. 
\end{gather}
Here we briefly summarize the scheme developed recently in \cite[Section~3]{barLud} to compute this integral with controlled accuracy. If we once again apply the rescaling $\eta = -1+\frac{2}{L}s_j$, then Eq.\ \eqref{eq:jf2} shows that the integrals we need to evaluate are of the form
\begin{equation}
\label{eq:nsing}
I(\V{y}) = \frac{L}{2}\int_{-1}^1 \frac{f(\eta)}{\norm{\V{X}^j(\eta) -\V{x}}^m} \, d\eta
\end{equation}
for $m=1, 3, 5$ and a smooth density $f(\eta)$. Now, the idea of \cite{barLud} is to extend the representation of $\V{X}^j(\eta)$ to the \textit{complex} plane and compute the complex root of $\norm{\V{X}^j\left(\eta^*\right)-\V{x}}=0$. Because the centerline representation $\V{X}^j(\eta)$ is available as a Chebyshev series, it is simple to solve for the root $\eta^*$ via Newton iteration. 

Once the root is known, the singularity can be removed from the integrand by rewriting Eq.\ \eqref{eq:nsing} as
\begin{equation}
\label{eq:intfactored}
I(\V{y}) = \frac{L}{2}\int_{-1}^1 \frac{f(\eta)((\eta-\eta^*)\overline{(\eta-\eta^*)})^{m/2}}{\norm{\V{X}^j(\eta) -\V{x}}^m} \frac{1}{((\eta-\eta^*)\overline{(\eta-\eta^*)})^{m/2}}\, d\eta, 
\end{equation}
so that the integral is written as a smooth function times a near singular function. Note the analogy with the finite part case, Eq.\ \eqref{eq:Jrewrite}, where the integrand is again written as the product of a smooth function times a singular function. In addition, the Bernstein radius of the root $\eta^*$ can be used to determine whether direct quadrature with $N$ points will suffice to compute the integral in Eq.\ \eqref{eq:intfactored} for any given tolerance \cite{barLud}.

The procedure of Section \ref{sec:tornFP} can be repeated to compute the integral via a monomial expansion. Let us define 
\begin{equation}
\label{eq:phi}
\phi(\eta)  = \frac{f(\eta)((\eta-\eta^*)\overline{(\eta-\eta^*)})^{m/2}}{\norm{\V{X}^j(\eta) -\V{x}}^m}
\end{equation}
and use the adjoint method of Eq.\ \eqref{eq:specscheme} to compute monomial coefficients of $\phi(\eta)$. The integrals 
\begin{equation}
q_k(\eta^*) = \frac{L}{2}\int_{-1}^1 \eta^k \frac{1}{((\eta-\eta^*)\overline{(\eta-\eta^*)})^{m/2}}\, d\eta,
\end{equation}
are then computable exactly by recurrence relations, as discussed in \cite[Section~3.1]{barLud}. Thus the entire accuracy of the scheme depends on representing $\phi(\eta)$ in Eq.\ \eqref{eq:phi} in a monomial basis accurately. Because of the scarcity of information on this, we have conducted our own tests in Appendix \ref{sec:nearfibtests}. These tests lead naturally to a method to compute the integral in Eq.\ \eqref{eq:jf1} to a guaranteed 3 digits of accuracy for any target and fiber. 

Appendix \ref{sec:nearfibtests} also shows how we can estimate the distance between a given target and fiber centerline to within 10\% accuracy. Suppose now that this distance, defined as $d$, is known exactly. Because each point on the fiber centerline is positioned inside a cross section of radius $r=\epsilon L$, the integral in Eq.\ \eqref{eq:jf1} only makes physical sense when $d > 2\epsilon L$ (when the ``cross sections'' are not in contact). In order to establish a smooth velocity field, we draw an analogy with \cite{ts04} to establish a continuous velocity field when $d=\mathcal{O}(\epsilon L)$. This choice is essentially arbitrary and is important solely to make the nonlocal mobility $\M{M}_{NL}$ continuous. Because of this, we have relegated it to Appendix \ref{sec:contvel}. 


\subsection{Fast summation \label{sec:ewald}}
Our goal in this section is to show how we can accelerate the sums over fibers that appear in Eq.\ \eqref{eq:fibevcont} so that they are linear in time. Eq.\ \eqref{eq:fibevcont} gives the velocity contribution at point $\V{X}^i(s_i)$ due to fibers $\V{X}^j(s_j)$ as 
\begin{equation}
\label{eq:otherfibs}
\left(\M{M}_{J}\V{f}\right)(s_i) := \frac{1}{8\pi\mu}\sum_{j \neq i} \int_0^L \left(\frac{\V{I}+\hat{\V{R}}(s_i,s_j)\hat{\V{R}}(s_i,s_j)}{\norm{\V{R}(s_i,s_j)}} + (\epsilon L)^2 \frac{\V{I}-3\hat{\V{R}}(s_i,s_j)\hat{\V{R}}(s_i,s_j)}{\norm{\V{R}(s_i,s_j)}^3}\right)\V{f}^j(s_j) \, ds_j,
\end{equation}
where $\V{R}=\V{X}^i(s_i)-\V{X}^j(s_j)$. Now let us discretize each of the integrals in Eq.\ \eqref{eq:otherfibs} using a quadrature scheme with $N$ points. Then we have 
\begin{equation}
\label{eq:otherfibs2}
\left(\M{M}_{J}\V{f}\right)(s_i) := \frac{1}{8\pi\mu}\sum_{j \neq i}\sum_{j=1}^N \left(\frac{\V{I}+\hat{\V{R}}(s_i,s_j)\hat{\V{R}}(s_i,s_j)}{\norm{\V{R}(s_i,s_j)}} + (\epsilon L)^2 \frac{\V{I}-3\hat{\V{R}}(s_i,s_j)\hat{\V{R}}(s_i,s_j)}{\norm{\V{R}(s_i,s_j)}^3}\right)\V{f}^j(s_j) \, w_j. 
\end{equation}
Temporarily disregarding the fact that this quadrature will not be accurarate for fibers sufficiently close to the target, we observe in Eq.\ \eqref{eq:otherfibs2} that we simply need to compute the velocity at point $\V{X}^i(s_i)$ due to a collection of other points positioned at $\V{X}^j(s_j)$. Because of the quadrature scheme, these points each have an associated ``force'' of $\V{f}^j(s_j)w_j$. Finally, the matrix kernel in the integrand of Eq.\ \eqref{eq:otherfibs2} is equivalent to the RPY kernel \cite{rpyOG, PSRPY} for spheres when the sphere radius $a=\sqrt{\frac{3}{2}}\epsilon L$. Our problem therefore reduces to using the RPY kernel to \textit{calculate the velocity at a collection of points due to forces at those points.} This well-studied problem can be treated with a number of fast algorithms. 

Because we consider periodic BCs in general here, we choose Ewald summation to handle naive quadratic-complexity sums in linear time. We begin this section by describing Ewald splitting on a sheared periodic domain for a system of points. We then describe how we correct Eq.\ \eqref{eq:otherfibs2} for fibers that give near singular integrands. 

\subsubsection{Sheared coordinate system}
In order to implement a shear flow in periodic boundary conditions, a strained coordinate system is necessary. We assume (without loss of generality) that $x$ is the flow direction, $y$ is the gradient direction, and $z$ is the vorticity direction. 
Let the total strain be $g(t)$. Then we define a strained coordinate system with axes
\begin{equation}
\label{eq:axes}
\V{e}_{x'} = \V{e}_x \qquad \V{e}_{y'} = \V{e}_y+g(t)\V{e}_x \qquad \V{e}_{z'} = \V{e}_z,
\end{equation}
and strained wave numbers
\begin{equation}
\label{eq:swnums}
k'_x=k_x \qquad k'_y = k_y+g(t)k_x \qquad k'_z = k_z.
\end{equation}
Here $k_x, k_y, k_z$ are the wave numbers when the periodicity is over the $x,y,$ and $z$ directions, while $k_x', k_y', k_z'$ are the wave numbers when the periodicity is over the $x',y',$ and $z'$ directions. 

The transformation between the two coordinate systems is given by $x\V{e}_x+y\V{e}_y+z\V{e}_z=x'\V{e}'_x+y'\V{e}'_y+z'\V{e}'_z$ where
\begin{gather}
\label{eq:primes}
x'=x-g(t)y \qquad y'=y \qquad z'=z \\[2 pt]
x=x'+g(t)y' \qquad y=y' \qquad z=z'. 
\end{gather}
In Eq.\ \eqref{eq:primes}, $x'$, $y'$, and $z'$ are all periodic on $[0,L]$. 

Now we use Eq.\ \eqref{eq:primes} to determine that 
\begin{equation}
\frac{\partial}{\partial x} = \frac{\partial}{\partial x'} \qquad \frac{\partial}{\partial y} = \frac{\partial}{\partial y'}-g(t)\frac{\partial}{\partial x'} \qquad \frac{\partial}{\partial z} = \frac{\partial}{\partial z'}. 
\end{equation}

We therefore have the Laplacian in the transformed space as \cite{moto11}
\begin{equation}
\Delta = \left(\frac{\partial^2}{\partial x'^2}+\left(\frac{\partial}{\partial y'}-g(t)\frac{\partial}{\partial x'}\right)^2+\frac{\partial^2}{\partial z'^2}\right). 
\end{equation}
In Fourier space, $\hat{\Delta} = \V{k}' \cdot \V{k}'$, where 

\begin{equation}
\label{eq:kprime}
\bm{k}'= (k_x',k_y'-g(t)k_x',k_z'), 
\end{equation}

Using Eq.\ \eqref{eq:swnums}, it is easy to see that $k':=\norm{\bm{k}'}=\norm{(k_x,k_y,k_z)}:=k$. It follows that we can simply replace $k$ in isotropic Fourier calculations by $k'$. 

\subsubsection{Ewald splitting for blobs \label{sec:ewblob}}
In this section, we implement the Ewald splitting of \cite{PSRPY} for the RPY tensor on a periodic domain. The idea of Ewald splitting or Ewald summation is to split the mobility matrix into a smooth long-ranged part and a non-smooth short-ranged part. The smooth ``far field'' part is done by standard Fourier methods, and the non-smooth ``near field'' part is truncated so that it is nonzero for $\mathcal{O}(1)$ neighbors per point. 

Applying this to the RPY kernel, the periodic RPY tensor for a blob with hydrodynamic radius $a$ can be written on the sheared domain as
\begin{equation}
\hat{\M{M}}(\V{x}_i',\V{x}_j')=\frac{1}{V\mu}\sum_{\small{\bm{k}'\neq \bm{0}}} e^{i\V{k}' \cdot (\V{x}'_i-\V{x}'_j)} \frac{1}{k'^2}\left(\bm{I}-\hat{\bm{k}'}\hat{\bm{k}'}^T\right)\text{sinc}^2\left(k'a\right).  
\end{equation} 
We next apply the screening function of Hasimoto \cite{Hsplit},
\begin{equation}
\label{eq:HspE}
\hat{H}(k',\xi)=\left(1+\frac{k'^2}{4\xi^2}\right)e^{-k'^2/4\xi^2}, 
\end{equation}
to split the mobility $\hat{\M{M}}$ into a far field and near field, given in Fourier space respectively by
\begin{equation}
\hat{\M{M}}^f (\V{x}_i',\V{x}_j')= \hat{\M{M}} (\V{x}_i',\V{x}_j') H(k',\xi) \qquad \hat{\M{M}}^n (\V{x}_i',\V{x}_j') = \hat{\M{M}} (\V{x}_i',\V{x}_j') (1-H(k',\xi)). 
\end{equation}
Here $\xi$ is a splitting parameter that controls the decay of the far field in Fourier space and near field in real space. Assuming that the near field decays rapidly enough that Fourier series can be replaced by Fourier integrals, the near field mobility can be computed in real space by inverse transforming its Fourier space representation,
\begin{equation}
\label{eq:nearRPY}
\bm{M}^n(\V{x}_i',\V{x}_j')=F(r,\xi)\left(\bm{I}-\hat{\bm{r}}\hat{\bm{r}}^T\right)+G(r,\xi)\hat{\bm{r}}\hat{\bm{r}}^T, 
\end{equation}
where $\V{r}=(\bm{x}_i-\bm{x}_j)_p$, $r=\norm{\bm{r}}$, and $p$ denotes the nearest periodic image in the slanted domain. While the nearest image is over the slanted domain, $\V{r}$ and $r$ are measured in undeformed space so that distances are computed using the Euclidean metric. The exact forms of $F$ and $G$ are given in \cite[Appendix~A]{PSRPY}.

Moving on to the far field mobility, its full expression as a function of the sheared wave numbers is given by
\begin{equation}
\M{M}^f (\V{x}_i',\V{x}_j')=\frac{1}{\mu V}\sum_{\small{\bm{k}'\neq \bm{0}}}  e^{i\V{k}' \cdot (\V{x}'_i-\V{x}'_j)} \frac{1}{k'^2}\left(\bm{I}-\hat{\bm{k}'}\hat{\bm{k}'}^T\right)\text{sinc}^2\left(k'a\right)H(k',\xi). 
\end{equation}

For multiple interacting particles, we have that 
\begin{equation}
\V{U}(\V{x}'_i) = \frac{1}{\mu V}\sum_{\small{\bm{k}'\neq \bm{0}}} \sum_j \left(\M{M}^{f}(\V{x}_i',\V{x}_j')+\M{M}^{n}(\V{x}_i',\V{x}_j')\right)\V{F}(\V{x}'_j). 
\end{equation} 

The algorithm to compute the far field velocity at the set of blobs positioned at $\V{x}_i$ with forces $\V{F}_i$ is therefore
\begin{enumerate}
\item Compute the coordinates of each of the blobs in $(x',y',z')$ space; denote these as $\V{x}'_i$
\item Perform a type 1 (nonuniform to uniform) NUFFT to obtain the sum
\begin{equation}
\sum_j e^{-\V{k}' \cdot \V{x}'_j} \V{F}(\V{x}'_j).
\end{equation}
\item Multiply each wave number $\V{k}' \neq 0$ in Fourier space by 
\begin{equation}
\frac{1}{\mu V k'^2}\left(\bm{I}-\hat{\bm{k}'}\hat{\bm{k}'}^T\right)\text{sinc}^2\left(k'a\right)H(k',\xi), 
\end{equation}
and denote this intermediate result by $\V{u}(\V{k}')$. 
\item Perform a type 2 (uniform to nonuniform) NUFFT to obtain the sum 
\begin{equation}
\V{U}(\V{x}_i') = \sum_{\V{k}' \neq 0} e^{\V{k}' \cdot \V{x}'_i} \V{u}(\V{k}').
\end{equation}
\end{enumerate}
See \cite{barnettES} for more information on computing these sums efficiently. 

The near field velocity is then added to the far field velocity. The near field velocity is computed at each point by summing Eq.\ \eqref{eq:nearRPY} over neighboring points whose distance is less than a precomputed value $r^*$. We choose $r^*$ so that the velocity at $r^*$ due to a unit force in the $\hat{\V{r}}$ direction is less than $10^{-3}$. 

\subsubsection{Corrections to Ewald}
The use of Ewald splitting for \textit{blobs} to compute the non-local velocity brings about a few problems. First, the velocity from a fiber onto itself is included in the total Ewald velocity. But this velocity is treated with the free space RPY kernel (which knows only about spherical shapes), rather than the SBT kernel (which more accurately treats the fiber slenderness). A similar problem exists for close fibers. In this case, however, it is the quadrature scheme in Eq.\ \eqref{eq:otherfibs2} that gives incorrect results, as $N$ points are not enough to integrate a nearly singular kernel and special quadrature is usually required (see Appendix \ref{sec:nearfibtests} for the precise distances where direct quadrature breaks down).  

In both of these cases, the answer we obtain for the velocity at point $\V{X}^i(s_i)$ due to \textit{the nearest periodic copy of} fiber $\V{X}^j$ is computed incorrectly by discretizing into blobs. For this reason, we subtract the free space RPY kernel \cite{rpyOG}
\begin{gather}
\label{eq:RPY}
\M{M}_{RPY}(\bm{X}^i(s_i), \V{X}^j) = \sum_{j=1}^N \left(C_1\left(\norm{\bm{R}}\right)\bm{I}+C_2\left(\norm{\bm{R}}\right)\hat{\bm{R}}\hat{\bm{R}}^T\right) \bm{f}^j(s_j) w_j,
\end{gather}
where $\V{R}=\V{X}^i(s_i)-\V{X}^j(s_j)$ and
\begin{gather}
C_1(r) = \begin{cases} \frac{1}{r}+\frac{2a^2}{3r^3} & r > 2a\\[2 pt] \frac{4}{3a}-\frac{3r}{8a^2} & r \leq 2a\end{cases} \qquad
C_2(r) = \begin{cases} \frac{1}{r}-\frac{2a^2}{3r^3} & r > 2a\\[2 pt] \frac{4}{3a}-\frac{r}{82a^2} & r \leq 2a\end{cases}. 
\end{gather}
After subtracting the free space RPY kernel discretized with $N$ points, for fiber $i \neq j$ we then apply the special quadrature scheme of Section \ref{sec:specquad} and the decision tree of Fig.\ \ref{fig:algflow} to compute the velocity at target $\V{X}^i(s_i)$ due to fiber $\V{X}^j$, given in Eq.\ \eqref{eq:jf1}, to a guaranteed accuracy of 3 digits. The non-local contribution of fiber $i$ to itself (i.e. the finite part integral) is treated by the scheme discussed in Section \ref{sec:tornFP}. 





\subsection{Implementation}
We use a Python implementation. \cmt{More to come here as (a) we discuss what to put here and (b) the code is finalized}. 

\section{Numerical tests}
In this section, we present two numerical tests that demonstrate the improved accuracy and robustness of our algorithm. In Section \ref{sec:local}, we first consider a single relaxing fiber in free space. We directly compare our results using local drag to those of \cite{ehssan17} to demonstrate the improved spatial discretization of inextensibility and boundary conditions. We also show second-order temporal accuracy, which improves on the first-order scheme used in \cite{ehssan17}. Next, in Section \ref{sec:fallfibs}, we consider falling fibers in periodic boundary conditions and show that we maintain second-order accuracy in time even when non-local terms are included. 

\subsection{Single fiber with local drag \label{sec:local}}
In this section, we compare the output of our algorithm directly with that of \cite{ehssan17} for a fiber with initial tangent vector
\begin{equation}
\label{eq:Xst0}
\bm{X}_s(s,t=0) = \frac{1}{\sqrt{2}}\begin{pmatrix} \cos{\left(s^3 (s-L)^3\right)}\\[2 pt] \sin{\left(s^3(s-L)^3\right)}\\[2 pt] 1 \end{pmatrix}. 
\end{equation}
This choice of tangent vector satisfies the boundary conditions $\displaystyle \bm{X}_{ss}\left(s=0,L\right)=\bm{X}_{sss}\left(s=0,L\right)=\bm{0}$ and the inextensibility constraint. Because there is no analytical solution for its configuration, we obtain $\V{X}(t=0)$ by integrating Eq.\ \eqref{eq:Xst0} on the $N$ point grid using the Chebyshev integration matrix \cite{greengard1991spectral}. We do this, as opposed to integrating Eq.\ \eqref{eq:Xst0} to machine precision numerically, so that the Chebyshev expansions on an $N$ point grid of $\V{X}_s$ and $\V{X}$ are consistent at $t=0$. Because the algorithm of \cite{ehssan17} updates $\V{X}$ rather than $\V{X}_s$, they proceed in the latter fashion, and so much of the spcatial differences between the two algorithms will be visible even at $t=0$. 

Beginning with the tangent vector in Eq.\ \eqref{eq:Xst0}, we simulate the fiber relaxation until $t_f=0.01$, using $E=\mu=1$ for simplicity and $\epsilon=10^{-3}$. For this test only, we use ellipsodial fibers to facilitate comparison with \cite{ehssan17}, so that $c(s)=-\text{log}(\epsilon^2)$ in Eq.\ \eqref{eq:fibevcont}. 

In general, we will use an $L^2$ function norm to compute the differences between configurations throughout this section. Given two configurations $\bm{X}$ and $\bm{Y}$, the $L^2$ norm of their difference is defined as
\begin{equation}
\label{eq:Errort}
E[\bm{X},\bm{Y}](t) = \left(\int_0^L \norm{\bm{X}(t)-\bm{Y}(t)}^2 \, ds\right)^{1/2} = \left( \sum_{i=1}^{1000} \norm{\bm{X}(s_i,t)-\bm{Y}(s_i,t)}^2 \, w_i  \right)^{1/2}. 
\end{equation}
The final term denotes the fact that we upsample each configuration to a common type 2 grid of $1000$ points to measure the error. 

\subsubsection{Spatial error}
We begin with an analysis of the spatial errors in both algorithms. To isolate the spatial errors, we set $\Delta t =10^{-6}$, so that a given spatial discretization is temporally converged to at least 7 digits. We use as an exact solution the result of \cite{ehssan17} (the strong formulation) when $N=24$.   

Fig.\ \ref{fig:locspatial} shows the $L^2$ error over time for several different spatial discretizations. We observe first that our weak formulation errors are relatively constant throughout the simulation, and are generally maximal when $t=0$. This stems from our Chebyshev integration of the tangent vectors to obtain the positions, rather than the exact integration used in the strong formulation simulations. Considering now the maximal $L^2$ error over time, Fig.\ \ref{fig:locspatial} clearly shows that our discretization vastly improves spatial accuracy over that of \cite{ehssan17}. In particular, only $N=4$ points in our weak formulation are required to obtain the same spatial error as $N=12$ points in the strong formulation of \cite{ehssan17}. Furthermore, we obtain spatial errors below $10^{-4}$ when $N=12$ points are used, so that our results are $100$ times more accurate for the same amount of work. 


\begin{figure}
\centering 
\includegraphics[width=70mm]{LocalFigs/SpatialLocalPAPER.eps}
\caption{\label{fig:locspatial}Spatial errors in the relaxing fiber simulation. The $L^2$ norm error in position over time is shown for $N=4$ (dashed-dotted blue), $N=8$ (dotted red), and $N=12$ (solid yellow). The norm is defined in Eq.\ \eqref{eq:Errort}. We compare our results to the strong formulation when $N=12$ (dashed purple) and find that we obtain errors $\approx 100$ times smaller when $N=12$. } 
\end{figure}

\subsubsection{Temporal error}
We next consider the temporal error in our algorithm by using the trajectory with $\Delta t = 10^{-6}$ as an exact solution. We then simulate the fiber trajectory with timesteps $2, 4, 8, $ and $16 \times 10^{-6}$ and record the maximal $L^2$ error in time using Eq.\ \eqref{eq:Errort}. The results are given in Fig.\ \ref{fig:loctemporal} for $N=12$ and $N=24$. We see second-order convergence in time for both spatial discretizations. This is an improvement on the first-order discretization used in \cite{ehssan17}. 

\begin{figure}
\centering
\subfigure[$N=12$]{ 
\includegraphics[width=70mm]{LocalFigs/TemporalLocalPAPER.eps}}
\caption{\label{fig:loctemporal}Temporal errors in local drag simulation. For each spatial discretization, we use $\Delta t=10^{-6}$ as an exact solution for and measure the maximum $L^2$  error in position (defined in Eq.\ \eqref{eq:Errort}) over time. Using $\Delta t = 2, 4, 8, 16 \times 10^{-6}$ and $N=12$ (blue circles) and $N=24$ (orange squares), we observe second-order convergence in time. \cmt{Empirical orders from largest $\Delta t$ to smallest for $N=12$: 1.69, 2.20, 1.72, and for $N=24$: 1.93, 1.96, 2.10. Put these in a table?} } 
\end{figure}

\subsection{Falling fibers in a periodic domain \label{sec:fallfibs}}


\section{Application: cross-linked actin mesh}
\subsection{Cross-linking forces}
\subsection{Results}

\section{Conclusion}

\appendix

\section{Near fiber quadrature \label{sec:nearfibtests}}
The goal of this appendix is to provide some rationale about our choice of discretization for the special quadrature scheme. Following this, we also discuss the creation of a continuous velocity field that is similar to the approach taken in \cite{ts04}. 

\subsection{Choice of special quadrature discretization}
Given the novelty of the special quadrature scheme, we test its accuracy here in our context. To do this, we generate 100 different inextensbile filaments whose tangent vectors and force densities are represented by the same exponentially decaying Chebyshev series with $N=16$ coefficients (the decay is such that the last coefficient has value at most $10^{-4}$). We place 100 targets a distance $d$ in the normal direction from the fiber centerlines and compute the integrals in Eq.\ \eqref{eq:jf2} by direct quadrature to 10 digits and also using the special quadrature scheme of \cite{barLud} with varying fiber centerline discretizations. Our goal is to obtain 3 digits of accuracy irrespective of the fiber shape and distance from target to fiber. 

\subsubsection{Accuracy of direct quadrature}
Our first objective is to determine when to call the special quadrature routine, i.e. when the accuracy of direct quadrature breaks down. To do this, we non-dimensionalize the distance $d$ by the fiber length $L$, so that the targets are positioned a fixed $d/L$ from the fiber. We then integrate Eq.\ \eqref{eq:jf1} directly with $N=16$ and $N=32$ points. As shown in Fig.\ \ref{fig:dirquad}, the boundary at which we start getting less than 3 digits of accuracy is $d/L \approx 0.15$ for $N=16$ and $d/L \approx 0.06$ for $N=32$. So we need some form of special quadrature below $d/L=0.15$ when $N=16$ and below $d/L=0.06$ when $N=32$. 

\begin{figure}
\centering
\subfigure[$N=16$]{
\includegraphics[width=70mm]{LudQuadFigs/dirNL16.eps}
\includegraphics[width=70mm]{LudQuadFigs/dirNL16_ds.eps}}
\subfigure[$N=32$]{
\includegraphics[width=70mm]{LudQuadFigs/dirNL32.eps}
\includegraphics[width=70mm]{LudQuadFigs/dirNL32_ds.eps}}
\caption{Relative errors using direct quadrature for the slender body kernel of Eq.\ \eqref{eq:jf1}. We consider (a) $N=16$ and (b) $N=32$ points and vary the distance $d/L$ of the targets from the fiber. We see that we obtain 3 digits of accuracy using $N=16$ when $d/L \geq 0.15$. For $N=32$, the 3 digit threshold occurs around $d/L = 0.06$. }
\label{fig:dirquad}
\end{figure}

\subsubsection{Accuracy of special quadrature}
Given that special quadrature must be called at a fixed $d/L$, the question of what discretization to use remains. The need for a finer discretization at small distances arises because the function $\phi(\eta)$ in Eq.\ \eqref{eq:phi} becomes less smooth (and therefore less well-represented by a monomial basis) as $\V{x}$ approaches the fiber centerline. In our evaluation of special quadrature accuracy, we non-dimensionalize by the fiber radius, so that we are concerned with small values of $d/r=d/(\epsilon L)$. 

Fig.\ \ref{fig:specquad} clearly shows that we need to use 2 panels of 32 points when $d/\epsilon L < 8$. When $d/\epsilon L \geq 8$, we obtain 3 digits of relative accuracy using 1 panel of 32 points along the fiber. The transition to 1 panel of 16 points occurs at larger distances, specifically $d/\epsilon L \geq 60$ allows for 1 panel of 16 points to give 3 digits. The vital point here is that $d/\epsilon L \gg 1$ when we can use 1 panel of 16 points, and so at this large distance we are simply integrating the Stokeslet, as the doublet correction is insignificant. This means that the relevant non-dimensional distance is $d/L=0.06$. So the distance at which 1 panel of 16 points gives 3 digits using special quadrature is the same as the distance at which 1 panel of 32 points gives 3 digits using direct quadrature. This means that we can bypass special quadrature with 1 panel of 16 points and instead simply do the integral directly with 32 points for $0.15 > d/L > 0.06$. 

\begin{figure}
\centering
\subfigure[$d/\epsilon L=2$]{\includegraphics[width=0.3\textwidth]{LudQuadFigs/d4_2pan32.eps}}
\subfigure[$d/\epsilon L=8$]{\includegraphics[width=0.3\textwidth]{LudQuadFigs/d16_1pan32.eps}}
\subfigure[$d/\epsilon L=64$]{\includegraphics[width=0.3\textwidth]{LudQuadFigs/d60_1pan16.eps}}
\caption{Integrating the slender body kernel with $L=2$ and $\epsilon=10^{-3}$. We consider 100 fibers with 100 targets per fiber and plot the relative error over the targets for (a) $d/\epsilon L = 2$, (b) $d/\epsilon L = 8$, and (c) $d/\epsilon L = 60$, $d/L = 0.06$ (here since $d/\epsilon L \gg 1$, $d/L$ is the relevant non-dimensional distance). The independent variable in is the mean relative curvature on the fiber, $\norm{\bm{X}_{ss}}/C_{circ}$, where $C_{circ} =2\pi/(L\sqrt{2})$ is the curvature of a circular fiber with the same arclength. }
\label{fig:specquad}
\end{figure}

\subsection{Estimating distance from the fiber}
Having established that the spatial discretization needed to achieve 3 digits depends on the distance of the target from the fiber, we next define a measure of distance from the centerline. 

\subsubsection{Long distances}
Let us consider first long distances. When we are far from the fiber, we can sample the fiber locations at $N_u=16$ uniform points and compute the distance to the fiber as the discrete minimum over these points. 

Fig.\ \ref{fig:unifders} shows the relative errors in the fiber distance computation using $N_u=16$ uniform points along the fiber. At $d/L = 0.15$, we observe an over-estimation of at most 5\% in the distance using uniform points. For $d/L=0.06$, the over-estimation of at most 15\% in the distance is predictably larger (for comparison, using 16 Chebyshev points gives an error as large as 30\%). So for $N_u=16$, we conclude that we can estimate to 5\% accuracy whether or not a target is within $d/L=0.15$ of the fiber using the uniform point scheme. Likewise for $d/L=0.06$, we can estimate the distance to about 15-20\% accuracy. This means that we can use the uniform point sampling procedure to tell us whether to call the special quadrature routine. 

\subsubsection{Short distances}
Once we determine the special quadrature scheme is needed, we need to use a different approximation for distances close to the fiber that determines when to change the number of panels. Consider the root $\eta^*$ determined from the near-singular quadrature scheme with 1 panel. We write 
\begin{equation}
\label{eq:eta}
\eta_r=\begin{cases} \text{Re}(\eta^*) & -1 \leq \text{Re}(\eta^*) \leq 1\\[2 pt] -1 & \text{Re}(\eta^*) < -1 \\[2 pt]1 &  \text{Re}(\eta^*) > 1 \end{cases}
\end{equation} 
as an approximation to the rescaled (on $[-1,1]$) $\eta$ coordinate on the fiber centerline of the closest point. If $\bm{g}(\eta)=\bm{g}_r(\eta)+i\bm{g}_i(\eta)$ is the representation of the fiber centerline, extended analytically to the complex plane, then the approximate shortest distance to the fiber is $d_r=\norm{\bm{x}-\bm{g}_r(\eta_r)}$. Note that this gives an approximation to the distance even for points that are off the fiber in the tangential direction. 

Fig.\ \ref{fig:grters} shows that when $d/\epsilon L \leq 8$, we make at most a 10\% error in using $d_r$ as the normal distance from the fiber. Thus we can use $d_r$ to determine whether we need 1 panel of 32, 2 panels of 32, or if the distance is less than $2\epsilon L$ and special approximations need to be performed. 

\begin{figure}
\centering
\subfigure[Large distances - evenly spaced points]{\label{fig:unifders}
\includegraphics[width=0.45\textwidth]{LudQuadFigs/dUnifN16.eps}
\includegraphics[width=0.45\textwidth]{LudQuadFigs/dUnifN32.eps}}
\subfigure[Special quadrature]{\label{fig:grters}
\includegraphics[width=0.45\textwidth]{LudQuadFigs/groots_d4m3.eps}
\includegraphics[width=0.45\textwidth]{LudQuadFigs/groots_d16m3.eps}}
\caption{Computing the normal distances from the fiber two different ways. (a) Relative errors in computing the distance for $d/L = 0.15$ (left) and $d/L= 0.06$ (center) by resampling the curve at $N_u=16$ uniformly spaced points and replacing the continuous minimization problem with a discrete one. (b) The relative error in computing the distance using $d_r=\norm{\bm{x}-\bm{g}_r(t_r)}$ from the special quadrature scheme. The error is at most 10\% of the distance for $d/\epsilon L = 2, 8$. \cmt{Plots in (b) are very uninformative. Replace with box plots? those would show the outliers better.}  }
\end{figure}

\subsection{Continuous velocity field \label{sec:contvel}}
Given a target $\V{x}$ and fiber centerline $\V{X}(s)$, we have established that we can estimate the minimum distance $d$ from $\V{x}$ to $\V{X}(s)$. We recall that the integral in Eq.\ \eqref{eq:jf1} only makes physical sense when $d > 2\epsilon L$, as otherwise the ``cross sections'' of the target fiber and the source fiber overlap. Therefore if $d \leq 2 \epsilon L$, we set the velocity at $\V{x}$ to be exactly equal to the centerline velocity on the closest point $\V{X}(\eta_r)$, where $\eta_r$ is defined in Eq.\ \eqref{eq:eta}. As a consequence of the asymptotics of SBT which deal with a pointwise, rather than averaged, flow field, this velocity does not coincide exactly with the integral of Eq.\ \eqref{eq:jf1} at $d=2\epsilon L$. Because of this, we set the velocity to be equal to that given in Eq.\ \eqref{eq:jf1} only when $d > 4\epsilon L$. Between $d=2\epsilon L$ and $d=4 \epsilon L$, we linearly interpolate the velocity between the centerline velocity and the integrand in Eq.\ \eqref{eq:jf1}, which can be computed to controlled accuracy using special quadrature. This linear interpolation technique is similar to that of \cite{ts04}, except here it is dependent on the problem physics and not the spatial discretization. 

The tests of this section lead naturally to the ``near fiber'' algorithm in Fig.\ \ref{fig:algflow}. Note that this algorithm includes the error estimates from the calculation of distances. We assume that there are at least $N=16$ points on the fiber, so that when $d/L \geq 0.15$, the velocity computed from direct quadrature without upsampling is accurate to at least 3 digits. 

\begin{figure}[ht]
\centering
\includegraphics[width=0.8\textwidth]{LudQuadFigs/FlowChart.png}
\caption{Overall near field algorithm, where $d^*=d_r/(\epsilon L)$. We assume that there are at least $N=16$ points on the fiber, so that when $d/L \geq 0.15$, the velocity computed from direct quadrature without upsampling is accurate to at least 3 digits. }
\label{fig:algflow}
\end{figure}


\bibliographystyle{plain}

\bibliography{SlenderBib}

\end{document}

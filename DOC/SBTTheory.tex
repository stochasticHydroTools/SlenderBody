We are interested in modeling inextensible, slender fibers using slender body theory. Thus our task is to solve the PDE
\begin{equation}
\label{eq:fibevcont}
\frac{\partial \bm{X}}{\partial t} -\bm{U}_0(\bm{X})= \bm{M}\left(\bm{f}^{E}+\bm{\lambda}\right),
\end{equation}
where $\bm{M}$ is a mobility operator that comes from slender body theory and $\bm{U}_0$ is a background flow. $\bm{f}^E$ are the bending forces on the fiber, and $\bm{\lambda}=(T\bm{X}_s)_s$ are tensile forces that act as Lagrange multipliers enforcing inextensibility of the fiber. The fibers are assumed to be free, and are therefore subject to the boundary conditions, $\displaystyle \bm{X}_{ss}\left(s=0,L\right)=\bm{X}_{sss}\left(s=0,L\right)=\bm{0}$. 

\subsection{Slender body theory}
In slender body theory, the 3D slender fiber with aspect ratio $\displaystyle{\epsilon = \frac{r(L/2)}{L}}$ is reduced to a one dimensional curve along the fiber centerline via an asymptotic expansion in $\epsilon$. The slender body mobility operator is given by
\begin{equation}
\label{eq:Msbt}
\bm{M}[\bm{f}](s) =\frac{1}{8\pi \mu}\left( \bm{ \Lambda}[\bm{f}](s)+\bm{J}[\bm{f}](s)\right),
\end{equation}
where $\bm{f}$ is the force per length on the fiber. The first term on the right side of Eq.\ \eqref{eq:Msbt} is the local operator $\bm{\Lambda}$ given by
\begin{equation}
\label{eq:localsbt}
\bm{\Lambda}=-\log{(\epsilon^2 e)}(\bm{I}+\bm{X}_s \bm{X}_s) + 2(\bm{I}-\bm{X}_s \bm{X}_s). 
\end{equation}
Notice that, because the fiber is inextensible, $\bm{X}_s$ is assumed to be a unit vector, and $\bm{X}_s \bm{X}_s$ is an outer product. Using the $\mathcal{O}(\log{\epsilon^2})$ term of $\bm{\Lambda}$ in place of $\bm{M}$ is synonymous with the \textit{local drag model}. In addition, note that the definition of $\bm{\Lambda}$ in Eq.\ \eqref{eq:localsbt} gives a slender body theory that is $\mathcal{O}(\epsilon^2)$ accurate for \textit{ellipsoidal fibers}, that is, fibers whose radius is given by $r(s)=2\epsilon \sqrt{s(L-s)}$, where $s\in[0,L]$. For cylindrical fibers, this form of SBT is $\mathcal{O}(1)$ accurate, and a different form of $\bm{\Lambda}$ must be used to give the same asymptotic accuracy. In particular, for a general radius $r(s)=a(s)\epsilon L$, the local term is given by  
\begin{align}
\label{eq:c_t_general}
\bm{\Lambda} =\left(c\left(\frac{2s}{L}-1\right)(\bm{I}+\bm{X}_s\bm{X}_s) +  (\bm{I}-3\bm{X}_s\bm{X}_s)\right)\bm{f}(s)
\end{align}
where
\begin{equation}
c(t) = \log{\left(\frac{2(1-t^2)+2\sqrt{(1-t^2)^2+16\epsilon^2a(t)^2}}{4a(t)^2\epsilon^2}\right)}.
\end{equation}
For a cylindircal fiber $a(t)=1$. This expression is taken from \cite{morifree} and is chosen so that $c(t) < \infty$ at the fiber ends.

For either cylindrical or ellipsoidal fibers, the $\mathcal{O}(1)$ non-local integral operator $\bm{J}$ is given by
\begin{equation}
\label{eq:intsbt}
\bm{J}[f](s) = \int_0^L \left(\frac{\bm{I}+\hat{\bm{R}}(s,s')\hat{\bm{R}}(s,s')}{\norm{\bm{R}(s,s')}}\bm{f}(s') - \frac{\bm{I}+\bm{X}_s(s) \bm{X}_s(s)}{|s-s'|} \bm{f}(s) \right) ds',
\end{equation}
where $\bm{R}(s,s')=\bm{X}(s)-\bm{X}(s')$ and $\hat{\bm{R}}$ is the corresponding unit vector. We will put off study of the non-local integral until a later time. 

\textbf{For the rest of this report, unless otherwise indicated, we set $\bm{M}=\bm{\Lambda}$ in Eq.\ \eqref{eq:localsbt} to compare with the data received from Floren}, which is obtained using a linearly implicit spectral discretization based on Tornberg and Shelley \cite{ts04}.

\input{ForPaperTh.tex}

\subsection{Discrete weak formulation}
\input{DiscrWk.tex}

\subsection{Evolving the tangent vector $\V{X}_s$}

A good choice of normal vectors that has a clear physical meaning is based on the Euler angles $\theta$ and $\phi$ for the unit tangent vector $\V{X}_s$,
\begin{equation}
\V{X}_s(\theta(s,t),\phi(s,t)) = \begin{pmatrix} \cos{\theta} \cos{\phi}\\[2 pt] \sin{\theta} \cos{\phi} \\[2 pt] \sin{\phi} \end{pmatrix}.
\end{equation}
The choice of normal vectors that is always orthonormal to $\V{X}_s$ on the unit sphere is
\begin{equation}
\label{eq:nangles}
\V{n}_1 =  \begin{pmatrix} -\sin{\theta}\\[2 pt] \cos{\theta}\\[2 pt]0 \end{pmatrix} \qquad \V{n}_2 =  \begin{pmatrix} -\cos{\theta} \sin{\phi}\\[2 pt] -\sin{\theta} \sin{\phi} \\[2 pt] \cos{\phi} \end{pmatrix}. 
\end{equation}
The position of the fiber $\V{X}$ can be recovered by 
\begin{equation}
\label{eq:XfromXs}
\V{X}(s)= \V{X}(0)+\int_0^s \V{X}_s(\theta(s),\phi(s)) \, ds. 
\end{equation}

By the right-handedness of the coordinate system
\begin{equation}
\label{eq:Xsupdate}
\V{X}_{st} = g_1(s)\V{n}_1 + g_2(s) \V{n}_2 = \left(g_1(s)\V{n_2}-g_2(s)\V{n}_1\right) \times \V{X}_s:=\V{\Omega}\left(\V{X}_s, \V{\alpha}\right) \times \V{X}_s. 
\end{equation}
Therefore, $\V{\Omega}\left(\V{X}_s, \V{\alpha}\right)=g_1(s)\V{n_2}-g_2(s)\V{n}_1$ can be viewed as a rotational velocity for $\V{X}_s$ on the unit sphere. The notation here indicates that $\V{\alpha}$ determines $g_1$ and $g_2$ and therefore $\V{\Omega}$. We will use this to update the configuration in our temporal integration schemes by rotating the tangent vector $\V{X}_s$ and then computing $\V{X}$ using \eqref{eq:XfromXs}; this strictly enforces inextensibility.

This theory is of course predicated on the smoothness of $\V{n}_1(s)$ and $\V{n}_2(s)$. In Eq.\ \eqref{eq:nangles}, it is clear that this only holds if $\theta$ is single-valued at $\phi=\pi/2$. Otherwise the signs of $\V{n}_1$ and $\V{n}_2$ can oscillate. We therefore set $\theta \left(\phi=\pm \pi/2\right)=0$ (numerically: \texttt{theta(abs((abs(phi)-pi/2)) < 1e-12) =0}). 

\subsection{Thermal Fluctuations}

This section was written by Aleks.

Let us summarize the equations of motion in the $L_2$ weak form, which appears to be most suitable to stochastic evolution on a space of inextensible worm-like chains. We add here Brownian noise following \cite{FluctuatingFibers_Saintillan_PNAS,FluctuatingFibers_Saintillan_PF}. Although this and all other prior work we know of was not careful about subtleties of multiplicative noise and infinite-dimensional Fokker-Planck equations, I believe ``the physicists" have the right form for the noise covariance and are (most likely) only missing some stochastic drift terms. In this section, we focus on a single fiber and we will ignore the translational motion of the fiber, which means we will set $\V{U}= {\partial\V{X}}/{\partial t}(s=0,t) = \V{0}$ as this is the easy part.

In order to avoid constraints, we will write here the evolution for the unit tangent vector along the fiber $\V{X}_s(\theta(s,t),\phi(s,t)) = \left( \cos{\theta} \cos{\phi},\, \sin{\theta} \cos{\phi},\,  \sin{\phi} \right)$ instead of $\V{X}$, but the inextensibility constraint will be implicit in the dynamical equations which arise from enforcing it with a tension Lagrange multiplier. Recall that we can complete $\V{X}_s$ to an orthonormal triad with the two orthogonal vectors $\V{n}_1(s)$ and $\V{n}_2(s)$ given in \eqref{eq:nangles}. While one could use the spherical angles $\theta(s,t)$ and $\phi(s,t)$ to describe the configuration this poses a problem at special points (e.g., the North pole) and is not very natural, so we instead think of $\V{X}_s(s,t)$ as being a curve on the unit sphere $S_2$ that evolves in time. The actual configuration $\V{X}(s,t)$, which enters in the equations also, is computed by integration of $\V{X}_s(s,t)$ using \eqref{eq:XfromXs}.

Recall that the process begins by choosing a set of basis functions $\phi_k(s)$ for $L_2:[0,s]$, which in the discrete formulation we take to be Chebyshev polynomials but this is not required. The dynamics of the fiber tangent vector is given by
\begin{equation}
\label{eq:Xsupdate_fluct}
\V{X}_{st} = \left( \M{K} \V{\alpha}\right)(s) = \V{\Omega}(s,t) \times \V{X}_s, 
\end{equation}
where
\begin{equation}
\label{eq:Omega_fluct}
\V{\Omega}(s,t) = \V{\Omega}\left(\theta(s,t),\phi(s,t), \V{\alpha}(t)\right) = 
	\sum_k \left[ \alpha_{1k}(t) \phi_k(s) \V{n_2}(s) - \alpha_{2k}(t) \phi_k(s) \V{n}_1(s) + \dots \right],
\end{equation}
where the dots indicate missing stochastic drift terms that we don't yet have any idea about except that they should be proportional to $k_B T$.

The direction of motion $\bm{\alpha}(t)=(\alpha_{1k}(t),\alpha_{2k}(t))$ (here $k$ is an index counting the basis functions) in the tangent to the manifold of inextensible fibers, as well as the constraint force density $\V{\lambda}(s,t)$, are computed by solving the saddle-point system
\begin{equation}
\label{eq:saddleL2_fluct}
\begin{pmatrix}
-\bm{M} & \bm{K}\\[4 pt]
\bm{K}^* & \bm{0}
\end{pmatrix}
\begin{pmatrix} 
\bm{\lambda}\\[4 pt]
\bm{\alpha}\\[4 pt]
\end{pmatrix} =  \begin{pmatrix} 
\bm{M}\bm{L}\bm{X} + \sqrt{2 k_B T} \bm{M}^{\frac{1}{2}} \V{\zeta} \\[4 pt]
\bm{0}\end{pmatrix},
\end{equation}
where $\V{\zeta}(s,t)$ is white noise in space and time. The specific form of the basis functions enters in the operator $\bm{K}$ defined via \eqref{eq:Xsupdate_fluct}, and its $L_2$ adjoint $\bm{K}^*$. Note that for the local drag model $\bm{M}=\bm{\Lambda}/(8\pi\mu)$, an explicit formula for the Brownian force density can be written (personal communication with David Saintillan):
\begin{equation}
\V{f}^B(s,t) = \sqrt{\frac{2 k_B T}{8\pi\mu}} \left(\bm{\Lambda}^{\frac{1}{2}} \V{\zeta}\right)(s,t) =  
	\sqrt{\frac{2 k_B T}{8\pi\mu}} \left( c_1(s) \M{I} + c_2(s) \V{X}_s \V{X}_s \right) \V{\zeta}(s,t),
\end{equation}
where the form of $c_1(s)$ and $c_2(s)$ can be computed by computing the covariance of $\V{f}^B$ and comparing it against \eqref{eq:c_t_general}. But this form should not be required in the formulation of the algorithm, which should also work even when the mobility includes non-local parts (which should be symmetric positive-definite, though this is not obvious just by looking at \eqref{eq:intsbt}).

Recall that $\bm{L}\V{X}=-E \V{X}_{ssss}$, taking into account the boundary conditions $\V{X}_{ss}=\V{X}_{sss}=0$ at both free ends. It is not clear how to handle the BCs in the context of fluctuating fibers. Should we put BCs somehow in the set of basis functions or should we enforce them as linear constraints on $\V{\alpha}$ as we (sort of) do in the numerical method? My feeling is that probably to get a working \emph{discrete} stochastic formulation in the end we will need to switch to a Galerkin formulation instead of the collocation approach, that is, we would want to replace \eqref{eq:veleq} with its Galerkin projection onto the space spanned by the basis functions. In this case we would enforce the BCs in the Galerkin formulation directly when integrating by parts, which imposes them weakly and not strictly, as seems the right thing to do. It is not clear, however, whether we also need to switch to Galerkin at the semi-formal level of the continuum $L_s$ formulation as well.

What we want from the equilibrium stochastic dynamics is that it be time-reversible with respect to a specific ``worm-like chain" invariant measure. This invariant measure for the tangent vector $\V{X}_s(s)$ is formed by the set of Brownian motions on $[0,L]$ on the unit $S_2$ sphere, where the diffusion coefficient of the Brownian motion is proportional to $k_B T / E$ (note that this means we cannot set $E=0$ and still make sense of this measure). This is more or less all we know, and now our goal is to figure out what to put in for the dots in \eqref{eq:Omega_fluct} to make this be so. The expectation is (but this should also be studied) is that if there is an additional conservative force density $\V{f}^{(c)}=-\delta{E^{(c)}}/\delta{\V{X}}$ other than bending, the dynamics would be time-reversible with respect to the same measure but weighted by the Boltzmann factor $\exp\left(-E^{(c)}\left[\V{X}\right]/\left(k_B T\right)\right)$.

One approach would be to represent Brownian motion on the unit $S_2$ sphere in some basis (say KL expansion), then truncate to get a smooth $\V{X}_s(s)$, and work with this as a function. Or maybe everything should be discretized to turn into a finite-dimensional Langevin equation (but observe this equation is very complicated in the end). These are still wide open questions that require some thought and discussion with Eric and Miranda.

